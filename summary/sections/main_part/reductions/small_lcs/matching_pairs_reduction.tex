\subsubsection{Matching Pairs Reduction}
\label{sec:small_lcs:matching_pairs_reduction}

We first present why $\alpha_M = 2$ was necessary in the previous reduction.
Let $(x,y)$ be the constructed \lcs{} instance.
Per definition of matching pairs, we know $M(x,y) \geq \numSymbols{0}{x} \cdot \numSymbols{0}{y}$.
We further have $\bigO{A}$ $O^\gamma$-blocks in $x$.
From \autoref{thm:small_lcs_construction} and \ref{thm:nvg} we know $\gamma = \Theta(D)$. 
Hence, $\numSymbols{0}{x} \geq \bigO{AD}$.
In $y$ we have two $0^{A\gamma'}$-blocks, so $\numSymbols{0}{y} \geq A\gamma' \geq \bigO{AD}$. %where we used $\gamma' = \Theta(D)$.
In total we have $M(x,y) \geq \bigO{(AD)^2} = \bigO{n(x,y)^2}$ and therefore $\alpha_M \geq 2$.
However, there is a parameter setting with $\alpha_M < 2$, e.g., for $\Sigma = \{ \sigma_1, \ldots, \sigma_n \}$ we can define $x = y := \sigma_1 \cdots \sigma_n$.
In this case, we have $\alpha_m = \alpha_L = \alpha_\Sigma = \alpha_d = 1$, $\alpha_\delta = \alpha_\Delta = 0$ and $\alpha_M = 1 < 2$.


To reduce our matching pairs, we can look at the parameter relation from \autoref{sec:fin_param_rels}, specifically \autoref{thm:matching_pairs_lb}.
Here we have shown that $M \geq \frac{L^2}{|\Sigma|}$.
Hence, we have the chance to reduce $M$ by increasing the alphabet size.
Therefore, the authors split the input vector sets into multiple partitions.
They then used the previous reduction on each partition and shift to disjoint alphabets.
Carefully combining these constructed strings remains the correctness while the used alphabet size increases.
Hence, $M$ decreases for the constructed instances.
For the full construction and proof we refer to the original paper \cite[Section 9.1.3]{Bringman.2018}.
