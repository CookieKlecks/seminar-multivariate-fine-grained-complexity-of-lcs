\subsubsection{Matching Pairs Reduction}

Before we lift the assumption $\alpha_M = 2$, we present why it was necessary in the previous reduction.
Let $(x,y)$ be the constructed \lcs{} instance as before.
Per definition of matching pairs, we know $M(x,y) \geq \numSymbols{0}{x} \cdot \numSymbols{0}{y}$.
Looking at the construction, we have $\bigO{A}$ blocks of $O^\gamma$ in $x$.
From \autoref{thm:small_lcs_construction} and \ref{thm:nvg} we further know $\gamma \geq |\nvg{a_i}| = \Theta(D)$. 
Hence, $\numSymbols{0}{x} \geq \bigO{AD}$.
In $y$ we have one block $0^{A\gamma'}$ at the beginning of $y$, so $\numSymbols{0}{y} \geq A\gamma' \geq \bigO{AD}$ where we bounded $\gamma'$ similar to $\gamma$.
In total we have $M(x,y) \geq \bigO{(AD)^2} = \bigO{n(x,y)^2}$ and therefore $\alpha_M \geq 2$.
However, there is a parameter setting with $\alpha_M < 2$, e.g., for $\Sigma = \{ \sigma_1, \ldots, \sigma_n \}$ we can define $x = y := \sigma_1 \cdots \sigma_n$.
In this case, we have $\alpha_m = \alpha_L = \alpha_\Sigma = \alpha_d = 1$, $\alpha_\delta = \alpha_\Delta = 0$ and $\alpha_M = 1 < 2$.


To reduce our matching pairs, we can look at the parameter relation from \autoref{sec:fin_param_rels}, specifically \autoref{thm:matching_pairs_lb}.
Here we have shown that $M \geq \frac{L^2}{|\Sigma|}$.
Hence, we have the chance to reduce $M$ by increasing the alphabet size.
Therefore, the authors split the input vector sets into multiple partitions.
They then used the previous reduction on each partition.
They combine these constructed strings while shifting them to different alphabets.
This is done in such a way that correctness remains while the used alphabet size increases, hence $M$ decreases for the constructed instances.
For the full construction and proof we refer to the original paper \cite[Section 9.1.3]{Bringman.2018}.
