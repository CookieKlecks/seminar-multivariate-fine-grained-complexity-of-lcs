\begin{lemma}
\label{lem:1-2vs1:non-ortho-upper-bound}
If $L(x_i, y_j) = \rho_1$ for all $i \in \indexSet{P}, j \in \indexSet{Q}$, we have
\[ 
L(x,y) \leq Q\Gamma + (Q-N)\ell_y + N\rho_1
\]
\end{lemma}


\begin{proof}
%
We define $x =: z_1z_2 \cdots z_Q$ such that $L(x,y) = \sum_{j=1}^Q L(z_j, \gLargeLCS{y_j})$.
We say that $x_i$ is aligned with $y_j$ if and only if $z_j$ contains strictly more than half of the prefix $0^{\gamma_1}$ of $\gLargeLCS{x_i}$.
Note that every $x_i$ is assigned to at most one $y_j$.
We say that $y_j$ is $k$-aligned if $k$ different $x_{i_1}, \ldots, x_{i_k}$ are aligned with $y_j$.
For the following, we define $H(w) := 1^{\gamma_2}(01)^{\gamma_3}w1^{\gamma_3}$.

First consider any $0$-aligned $y_j$.
Per definition is $z_j$ a subsequence of $0^{\frac{\gamma_1}{2}}H(x_i)0^{\frac{\gamma_1}{2}}$.
It is possible to show that we can bound
\[
L(z_j, \gLargeLCS{y_j}) = L(0^{\frac{\gamma_1}{2}}H(x_i)0^{\frac{\gamma_1}{2}}, 0^{\gamma_1}H(y_j)) \leq \gamma_1 + \gamma_3 + \ell_y =: L_0.
\]
For a full proof of this, we refer to the original paper \cite[Lemma 9.6\footnote{Lemma 9.6 combines this and the following lemma in this summary. The proof of the claim is in the proof for the second inequality of Lemma 9.6.}]{Bringman.2018}.
But to give an intuition observe, that $y_1$ is chosen so big, that it is always optimal to match the full $0^{\gamma_1}$ block.
This match also stretches over $H(x_i)$.
The remainder $H(y_j)$ then can only match with remaining zeroes in the right $0^{\frac{\gamma_1}{2}}$ block.
With $\numSymbols{0}{H(y_j)} \leq y_3 + \ell_y$ we obtain the bound.


Due to appropriate $\gamma_1$, $\gamma_2$ and $\gamma_3$, one can further show that when $y_j$ is $1$-aligned we have
\[
	L(z_j, \gLargeLCS{y_j}) \leq \Gamma + L(x_i, y_j) = \Gamma + \rho_1 =: L_1,
\]
where we used the assumption that $L(x_{i'}, y_{j'}) = \rho_1$ for every $i', j'$.
For a full proof we again refer to the original paper.
%
Further when $y_j$ is $k$-aligned for $k \geq 2$, we can trivially bound 
\[
L(z_j, \gLargeLCS{y_j}) \leq |\gLargeLCS{y_j}| = \Gamma + \ell_y =: L_k.
\]

Let now $n_0$, $n_1$ and $n_k$ be the number of $0$-, $1$- and $k$-aligned $y_j$ ($k \geq 2$).
Because ever $y_j$ has exactly one alignment category, we have $n_0 + n_1 + n_k = Q$.
Further $2n_k + n_1 \leq P$ as every $k$-aligned $y_j$ is aligned with $k$ unique $x_i$s and every $x_i$ with at most one $y_j$.
By the previous bounds, we obtain
\begin{equation}
\label{eq:1-2vs1:non-ortho-upper-bound:eq-ns}
	L(x,y) = \sum_{j=1}^Q L(z_j, \gLargeLCS{y_j} \leq n_0 L_0 + n_1 L_1 + n_k L_k 
\end{equation}
It remains to show that this is bounded by $Q\Gamma + (Q-N)\ell_y + N\rho_1$. 
For this we show the following claim.
\begin{claim}
Setting $n_0 = \hat{n}_0 := 0$, $n_1 = \hat{n}_1 := N$ and $n_k = \hat{n}_k := (Q-N)$ maximizes
\[
\mathcal{L}(n_0, n_1, n_k) := n_0 L_0 + n_1 L_1 + n_k L_k
\]
under the assumptions $n_0 + n_1 + n_k = Q$ and $2n_k + n_1 \leq P = 2Q - N$.
\end{claim}
Observe first that these specific values satisfy the assumptions.
Now assume $n_0 > 0$ would maximize $\mathcal{L}(n_0, n_1, n_k)$ for any $n_1$ and $n_k$.
If we define $n'_0 := n_0 - 1$, $n'_1 := n_1 + 2$ and $n'_k := n_k - 1$, we have
\[
	\mathcal{L}(n'_0, n'_1, n'_k) = \mathcal{L}(n_0, n_1, n_k) + 2L_1 - L_0 - L_k > \mathcal{L}(n_0, n_1, n_k),
\]
as $L_0 + L_k = \Gamma + \gamma_1 + \gamma_3 + 2\ell_y < \Gamma + \gamma_1 + \gamma_2 + \gamma_3 < 2\Gamma \leq 2L_1$.
That is a contradiction, hence, $n_0 = 0$ if we want to maximize $\mathcal{L}(n_0, n_1, n_k)$.
Now we obtain from the first assumption $n_1 \leq Q - n_k$.
Putting this into the second assumption, we obtain
\begin{align*}
2n_k + Q - n_k &\leq 2Q - N %
&\Leftrightarrow& %
&n_k \leq Q - N.
\end{align*}
Hence, $\mathcal{L}(n_0, n_1, n_k)$ is maximized for $n_k = Q-N$, because $L_k \geq L_1 \geq 0$. 
The claim follows.



Now we can further bound \autoref{eq:1-2vs1:non-ortho-upper-bound:eq-ns} by
\begin{align*}
L(x,y) \leq \hat{n}_0 L_0 + \hat{n}_1 L_1 + \hat{n}_k L_k &= NL_1 + (Q-N) L_k  \\
	&= N\Gamma + N\rho_1 + (Q-N)(\Gamma + \ell_y) \\
	&= Q\Gamma + (Q-N)\ell_y + N\rho_1
\end{align*}
\end{proof}




