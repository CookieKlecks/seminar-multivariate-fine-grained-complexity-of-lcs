\subsection{Small LCS}
For the first reduction, we assume $\alpha_\delta = \alpha_m$, i.e., the number of deleted symbols in the smaller string is asymptotically the same as its length.
This reduction is similar to one in a previous fine-grained analysis of \lcs{} \cite{Bringmann.2015}.
We first present a gadget to embed vectors into strings.
This gadget will also be used in the other reductions.
Then we show how to build the full \lcs{} instances from \ov{} instances by using several instances of these gadgets.



\begin{theorem}[Normalized Vector Gadget]
\label{thm:nvg}
For any two vectors $a$ and $b$ of dimension $D$, there are two strings $\nvg{a}$ and $\nvg{b}$ with length $\Theta(D)$ such that
\[
L(\nvg{a}, \nvg{b}) = \begin{cases}
		\rho_0 & if \langle a, b \rangle = 0\\
		\rho_1 & otherwise
	\end{cases}
\]
for an appropriate $\rho_0 > \rho_1$.
\end{theorem}

% Coordinate gadgets
\newcommand{\cgZeroX}{\mathbf{0_x}}
\newcommand{\cgOneX}{\mathbf{1_x}}
\newcommand{\cgZeroY}{\mathbf{0_y}}
\newcommand{\cgOneY}{\mathbf{1_y}}
\newcommand{\cg}[2]{\text{\textsc{Cg}}(#1,#2)}


% Vector Gadges
\newcommand{\guardX}[1]{G_X(#1)}
\newcommand{\guardY}[1]{G_Y(#1)}
% example:
\newcommand{\exVGguardOne}[3]{\node[base, number, guard] (#3) at ($(#1) + (#2,0)$) {$1^{\gamma_2}$};}
\newcommand{\exVGguardZero}[3]{
		\node[base, number, guard] (#3) at ($(#1) + (#2,0)$) {$0^{\gamma_1}$};}
\newcommand{\exVGbigGuardZero}[3]{
		\node[base, number, big guard] (#3) at ($(#1) + (#2,0)$) {$0^{\gamma_3}$};}
\newcommand{\exVGendGuardY}[3]{
		\node[base, number, big guard] (#3) at ($(#1) + (#2,0)$) {$0^{n\gamma_4}$};}
		
		
\newcommand{\vg}[1]{\text{\textsc{Vg}}(#1)}


\newcommand{\vgWithDummy}[7]{%
	\node[cell] (#1_1) at (base) {#2};
	\node[cell] (#1_2) at ($(base) + (1,0)$) {#3};
	\node[cell] (#1_3) at ($(base) + (2,0)$) {#4};
    \node (#1_dots) at ($(base) + (3,0)$) {$\cdots$};
    \node[cell] (#1_4) at ($(base) + (4,0)$) {#5};
    \node[cell] (#1_dummy) at ($(base) + (5,0)$) {#6};
        
    \node[] (#1_vector) at ($(base) + (2,0) #7 (0,12pt)$) {coordinates};
        
    \coordinate (#1_line_top) at ($(#1_dummy.west) + (0,14pt)$);
    \coordinate (#1_line_bot) at ($(#1_dummy.west) - (0,14pt)$);
    \path[draw, thick] (#1_line_top) -- (#1_line_bot);
        
    \node[] (#1_dummy_text) at ($(base) + (5.2,0) #7 (0,12pt)$) {};
    
		\begin{pgfonlayer}{mid}
    \node[box vg, fit=(#1_1) (#1_vector) (#1_dummy_text) (#1_line_top.north) (#1_line_bot.south)] (#1_box) {};
    \end{pgfonlayer}
}

\newcommand{\vgDummyS}[6]{%
	\node[cell] (#1_1) at (base) {#2};
	\node[cell] (#1_2) at ($(base) + (1,0)$) {#3};
    \node (#1_dots) at ($(base) + (2,0)$) {$\cdots$};
    \node[cell] (#1_4) at ($(base) + (3,0)$) {#4};
    \node[cell] (#1_dummy) at ($(base) + (4,0)$) {#5};
        
    \node[] (#1_vector) at ($(base) + (1.5,0) #6 (0,12pt)$) {zero vector};
        
    \coordinate (#1_line_top) at ($(#1_dummy.west) + (0,14pt)$);
    \coordinate (#1_line_bot) at ($(#1_dummy.west) - (0,14pt)$);
    \path[draw, thick] (#1_line_top) -- (#1_line_bot);
        
    \node[] (#1_dummy_text) at ($(base) + (4.2,0) #6 (0,12pt)$) {};
    
		\begin{pgfonlayer}{mid}
    \node[box vg, fit=(#1_1) (#1_vector) (#1_dummy_text) (#1_line_top.north) (#1_line_bot.south)] (#1_box) {};
    \end{pgfonlayer}
}

\tikzset{
    cell/.style = {
        draw, fill=white, rounded corners=0, rectangle,
        minimum width=2em, minimum height=14pt,
        anchor=center, inner sep=1pt
    },
    box vg/.style = {
        draw, fill=lightgray, inner sep=5pt, rounded corners
    },
    box nvg/.style = {
        draw, fill=lightgray!60!white, inner sep=5pt, rounded corners
    },
    good edge/.style = {
        draw, color=teal
    },
    bad edge/.style = {
        draw, color=orange
    }
}

\begin{figure}

    \centering
    % First subfigure
    \begin{subfigure}[b]{0.45\textwidth}
        \centering
\begin{tikzpicture}[
    box/.style={draw, rounded corners, fill=blue!10, inner sep=3pt},
    edge/.style={green!70!black, thick},
    labelstyle/.style={font=\scriptsize},
    x=1.6em
]

\coordinate (base) at (0,2);
\vgWithDummy{a}{$\cgZeroX$}{$\cgOneX$}{$\cgZeroX$}{$\cgOneX$}{$\cgZeroX$}{+}
\node[above=2pt of a_box.north] (vgai) {$\vg{a_i}$}; 

\coordinate (base) at ($(a_dummy.east) + (1.5,0)$);
\vgDummyS{s}{$\cgZeroX$}{$\cgZeroX$}{$\cgZeroX$}{$\cgOneX$}{+}
\node[above=2pt of s_box.north] (S) {$S$}; 


\begin{pgfonlayer}{bg}
    \node[box nvg, draw, fit=(a_box) (s_box) (vgai) (S)] (nvgai) {};
    \node[above=2pt of nvgai.north] {$\nvg{a_i}$}; 
\end{pgfonlayer}
 




\path ($(a_1.south)!0.5!(s_1.south)$) ++(0,-2.2) coordinate (base);

\vgWithDummy{b}{$\cgZeroY$}{$\cgOneY$}{$\cgOneY$}{$\cgOneY$}{$\cgOneY$}{-}

\vgWithDummy{b}{$\cgOneY$}{$\cgZeroY$}{$\cgOneY$}{$\cgZeroY$}{$\cgOneY$}{-}

\node[below=2pt of b_box.south] (vgbj) {$\vg{b_j}$}; 

\begin{pgfonlayer}{bg}
    \node[box nvg, draw, fit=(b_box) (vgbj)] (nvgbj) {};
    \node[below=2pt of nvgbj.south] {$\nvg{b_j}$}; 
\end{pgfonlayer}


\draw[good edge] (a_1.south) edge node[fill=white] {$\rho_0$} (b_1.north);
\draw[good edge] (a_2.south) edge node[fill=white] {$\rho_0$} (b_2.north);
\draw[good edge] (a_3.south) edge node[fill=white] {$\rho_0$} (b_3.north);
\draw[good edge] (a_4.south) edge node[fill=white] {$\rho_0$} (b_4.north);
\draw[good edge] (a_dummy.south) edge node[fill=white] {$\rho_0$} (b_dummy.north);

\end{tikzpicture}
        \caption{\lcs{} visualization when $\langle a_i, b_j \rangle = 0$.}
        \label{fig:nvg:ortho}
    \end{subfigure}
    \hfill
    % Second subfigure
    \begin{subfigure}[b]{0.45\textwidth}
        \centering
\begin{tikzpicture}[
    box/.style={draw, rounded corners, fill=blue!10, inner sep=3pt},
    edge/.style={green!70!black, thick},
    labelstyle/.style={font=\scriptsize},
    x=1.6em
]

\coordinate (base) at (0,2);
\vgWithDummy{a}{$\cgZeroX$}{$\cgOneX$}{$\cgZeroX$}{$\cgOneX$}{$\cgZeroX$}{+}
\node[above=2pt of a_box.north] (vgai) {$\vg{a_i}$}; 

\coordinate (base) at ($(a_dummy.east) + (1.5,0)$);
\vgDummyS{s}{$\cgZeroX$}{$\cgZeroX$}{$\cgZeroX$}{$\cgOneX$}{+}
\node[above=2pt of s_box.north] (S) {$S$}; 


\begin{pgfonlayer}{bg}
    \node[box nvg, draw, fit=(a_box) (s_box) (vgai) (S)] (nvgai) {};
    \node[above=2pt of nvgai.north] {$\nvg{a_i}$}; 
\end{pgfonlayer}
 




\path ($(a_1.south)!0.5!(s_1.south)$) ++(0,-2.2) coordinate (base);

\vgWithDummy{b}{$\cgZeroY$}{$\cgOneY$}{$\cgOneY$}{$\cgOneY$}{$\cgOneY$}{-}

\vgWithDummy{b}{$\cgOneY$}{$\cgZeroY$}{$\cgOneY$}{$\cgZeroY$}{$\cgOneY$}{-}

\node[below=2pt of b_box.south] (vgbj) {$\vg{b_j}$}; 

\begin{pgfonlayer}{bg}
    \node[box nvg, draw, fit=(b_box) (vgbj)] (nvgbj) {};
    \node[below=2pt of nvgbj.south] {$\nvg{b_j}$}; 
\end{pgfonlayer}


\draw[good edge] (s_1.south) edge node[fill=white] {$\rho_0$} (b_1.north);
\draw[good edge] (s_2.south) edge node[fill=white] {$\rho_0$} (b_2.north);
\draw[good edge] (s_dots.south) edge node[fill=white] {$\rho_0$} (b_3.north);
\draw[good edge] (s_4.south) edge node[fill=white] {$\rho_0$} (b_4.north);
\draw[bad edge] (s_dummy.south) edge node[fill=white] {$\rho_1$} (b_dummy.north);


\end{tikzpicture}
        \caption{\lcs{} visualization when $\langle a_i, b_j \rangle \neq 0$.}
        \label{fig:nvg:non-ortho}
    \end{subfigure}
    \caption{This figure shows the general idea of a normalized vector gadget. $S$ is hereby the normalization part that lower bounds the \lcs{} value for the case that the vectors are not orthogonal. Hence, the \lcs{} length remains the same no matter if they are just in one coordinate or all of them non-orthogonal.}
    \label{fig:nvg}

\end{figure}



We refer to the previous fine-grained analysis \cite{Bringmann.2015} for the proof.
See \autoref{fig:nvg} for a visualization of an \nvgName{}.
Note however, that we can now infer from the \lcs{} of two of the \nvgName{}s whether the respective vectors are orthogonal.
Because the \lcs{} only admits two specific values, we can also infer from the sum of \lcs{} lengths of multiple vector pairs whether any of these pairs are orthogonal.
We now use this fact to build the final construction.


% =============================== OR Gadget ========================
\begin{theorem}
\label{thm:small_lcs_construction}
	For an OV instance $\mathcal{A} = \{a_1, \ldots, a_A\}$ and $\mathcal{B} = \{b_1, \ldots, b_B\}$ with $A \geq B$ and dimension $D$, we can construct the strings
\begin{align*}
	x &=& \nvg{a_1}\ 0^\gamma\ \cdots\ 0^\gamma\ \nvg{a_A}\ &0^\gamma\ %
	\nvg{a_1}\ 0^\gamma\ \cdots\ 0^\gamma\ \nvg{a_A}\\
	y&=& 0^{A\gamma'}\ \nvg{b_1}\ 0^{\gamma}\ \nvg{b_2}\ &0^\gamma\ \cdots\ 0^\gamma\ \nvg{b_B}\ 0^{A\gamma'}
\end{align*}

in time $\bigO{AD}$ with large enough $\gamma, \gamma' \geq |\nvg{a_i}| + |\nvg{b_j}|$, satisfying:

\begin{enumerate}[(i)]
    \item\label{thm:small_lcs_construction:infer} $L(x,y) \geq \rho \Leftrightarrow \exists a_i, b_j:\; \langle a_i,b_j \rangle = 0$ \quad\quad (for some appropriate $\rho$)

    \item\label{thm:small_lcs_construction:size} $|x|, |y| = \bigO{AD}$
\end{enumerate}
\end{theorem}	


\newcommand{\sLCSnvg}[3]{%
	\node[base, number, input, minimum width=4.5em, #3] (#1) at (base) {$\nvg{#2}$};
}
\newcommand{\sLCSnvgWithGuard}[4]{%
	\sLCSnvg{#1}{#2}{#4}
	\node[base, number, guard] (#3) at ($(#1.east)$) {$0^\gamma$};
}


\newcommand{\sLCSOr}{%
		\tikzstyle{base} = [minimum width=1.5em, inner sep=1pt, anchor=west]
		\tikzstyle{text element} = []
		\tikzstyle{number} = [minimum height=14pt]  
		\tikzstyle{guard} = [fill=lightgray]  
		\tikzstyle{guard edge} = [color=lightgray]
		\tikzstyle{big guard} = [fill=lightgray!75!black]
		\tikzstyle{big guard edge} = [color=lightgray!75!black]
		\tikzstyle{input} = [fill=cyan]
		\tikzstyle{input edge} = [color=cyan]
		\tikzstyle{cell} = [draw, minimum width=1.5em, fill=white]
		\tikzstyle{skipped} = [fill = lightgray!75!white]
		
		\node[base, text element] (eqX) at (0,0) {$=$};
		\node[base, text element, anchor=east] (x) at (eqX.west) {$x$};
		
		\coordinate (base) at ($(eqX.east) + (0.5,0)$);
		\sLCSnvgWithGuard{a1}{a_1}{a_g1}{skipped}
		
		\node[base, minimum width=3em] (a_dots_1) at (a_g1.east) {$\cdots$};
		
		
		\node[base, number, guard] (a_gj_left) at (a_dots_1.east) {$0^\gamma$};
		\coordinate (base) at (a_gj_left.east);
		\sLCSnvgWithGuard{aj}{a_j}{a_gj}{}
		
		\node[base, minimum width=3em] (a_dots_2) at (a_gj.east) {$\cdots$};		
		\node[base, number, guard] (a_gjB_left) at (a_dots_2.east) {$0^\gamma$};
		
		\coordinate (base) at (a_gjB_left.east);
		\sLCSnvgWithGuard{ajB}{a_{j+B-1}}{a_gjB}{}
		
		\node[base, minimum width=3em] (a_dots_3) at (a_gjB.east) {$\cdots$};
		
		\node[base, number, guard] (a_gA_left) at (a_dots_3.east) {$0^\gamma$};
		\coordinate (base) at (a_gA_left.east);
		\sLCSnvg{aA}{a_A}{skipped}
		
		
		\node[base, text element] (eqY) at (0,-1.2) {$=$};
		\node[base, text element, anchor=east] (y) at (eqY.west) {$y$};
		
		
		\coordinate (base) at ($(eqY.east) + (0.5,0)$);
		\node[base, number, guard, minimum width=3em] (b_left) at (base) {$0^{A\gamma'}$};
		\coordinate (base) at (b_left.east);
		\sLCSnvgWithGuard{b1}{b_1}{b_g1}{}
		\coordinate (base) at (b_g1.east);
		\node[base, number, input, minimum width=6em] (b2) at (base) {$\nvg{b_{2}}$};
		\node[base, number, guard] (b_g2) at (b2.east) {$0^\gamma$};
		
		\node[base, minimum width=4.5em] (b_dots_1) at (b_g2.east) {$\cdots$};
		
		\coordinate (base) at (b_dots_1.east);
		\node[base, number, guard] (b_g3) at (base) {$0^\gamma$};
		\node[base, number, input, minimum width=6em] (b3) at (b_g3.east) {$\nvg{b_{B-1}}$};
		\node[base, number, guard] (b_g4) at (b3.east) {$0^\gamma$};
		\coordinate (base) at (b_g4.east);
		\sLCSnvg{bB}{b_{B}}{}
		\node[base, number, guard, minimum width=3em] (b_right) at (bB.east) {$0^{A\gamma'}$};
}

%\begin{figure}
%\begin{tikzpicture}
%	\sLCSOr{}
%	
%
%	\path[draw, guard edge] (a_g1.south) edge (b_left.north);
%	%\path[draw, guard edge] (a_dots_1.south) edge (b_left.north);
%	\path[draw, guard edge] (a_gj_left.south) edge (b_left.north);
%	
%	\path[draw, input edge] (aj.south) edge (b1.north);
%	
%	\path[draw, guard edge] (a_gj.south) edge (b_g1.north);
%	
%	\path[draw, input edge] (ajB.south) edge (b2.north);
%	
%	\path[draw, guard edge] (a_gjB.south) edge (b_g2.north);
%	%\path[draw, guard edge] (a_dots_2.south west) edge (b_g3.north);
%	
%	\path[draw, input edge] (a_dots_2.south) edge (b3.north);
%	\path[draw, guard edge] (a_gA_left.south) edge (b_g4.north);
%	\path[draw, input edge] (aA.south) edge (bB.north);
%\end{tikzpicture}
%\caption{Visualization of }
%\end{figure}

We will not prove this theorem here, but only provide an intuition for its correctness.
The blocks of $0^\gamma$ guard each normalized vector gadget such that for any \lcs{} every $\nvg{b_j}$ will align with an $\nvg{a_i}$ and will not match with multiple \nvgName{}s.
The blocks of zeroes at the start and end of $y$ allow to \enquote{skip} \nvgName{}s of $x$ to align the $B$ \nvgName{}s of $y$ with any consecutive subsequence of $B$ \nvgName{}s of $x$.
Because of the repeated structure of $x$, an \lcs{} can then \enquote{pick} an alignment where a desired $\nvg{a_i}$ and $\nvg{b_j}$ align, i.e., one where $\langle a_i, b_j \rangle = 0$ if existing.
We can then use $L(\nvg{a_i}, \nvg{b_j}) \in \{\rho_0, \rho_1\}$ to find an appropriate $\rho$ as decision boundary to decide the $\ov$ instance.
This should motivate the correctness of our construction.


\paragraph*{Reduction}
We now use the previous theorem to instantiate the actual reductions with the desired bounds.
Let $n \geq 1$ be arbitrary and consider any parameter setting $\alpha$ satisfying \autoref{tab:restrictions} and $\alpha_\delta = \alpha_m$.
We construct strings $x$ and $y$ as in \autoref{thm:small_lcs_construction} with $D = n^{o(1)}$, $A := \lfloor \frac{L}{D} \rfloor$ and $B := \lfloor \frac{m^2}{LD} \rfloor$.\footnote{In the original paper they used $B := \lfloor \frac{d}{LD} \rfloor$. We however simplified this, because we later restrict us to a reduced parameter space for simplicity reasons.}
Note that \uovh{} implies a lower bound of $(AB)^{1-o(1)} = n^{2\alpha_m - o(1)} = m^{2 - o(1)} = (\delta m)^{1-o(1)}$ where the last equality holds in the current case $\alpha_\delta = \alpha_m$.
By \autoref{thm:small_lcs_construction} (\ref{thm:small_lcs_construction:infer}) we can infer from $L(x,y)$ whether any vectors are orthogonal.
Further the running time of the reduction is $\bigO{AD} = \bigO{L} \leq \bigO{\delta m}$ since trivially $L \leq m$.
This proves a lower bound for any algorithm for $\lcsy{\mathbb{\alpha}}$ of $(\delta m)^{1-o(1)}$ assuming we span the full parameter space.
The latter part now remains to be shown, i.e., for what \lcs{} parameter combinations we can proof a conditional lower bound.



\paragraph*{Parameter Space}
For a multivariate analysis we want to have reductions that span the full (non-trivial) parameter space to show conditional lower bounds for any parameter combination.
Hence, our goal is now to show that given a target parameter setting $\mathbf{\alpha}$, we can create instances that are in $\lcsy{\mathbf{\alpha}}$.
We write $p = n^{\alpha_p}$ as the target value for any parameter $p \in \mathcal{P}$ and $p(x,y)$ as the actual value for the \lcs{} instance $(x,y)$.

The authors of the original paper have shown monotonicity of the parameter space \cite[section 4.2]{Bringman.2018}, i.e., it suffices to show that $p(x,y) \leq \bigO{p}$ for every possible target value of $p \in \mathcal{P}$.
Note that the monotonicity is not trivial and even fails for the case $|\Sigma| = 2$, but we do not handle this case here.

%Recall that we remain with the assumption $\alpha_\delta = \alpha_m$, i.e., $\bigO{\delta} = \bigO{m}$.

We will now show this using a reduced parameter set of $P' = \{n, m, L, \Sigma, M\}$ for the current case.
For a proof of the full parameter set we refer to the original paper \cite[section 9.1.2]{Bringman.2018}.
We have trivially and by definition $L(x,y) \leq m(x,y) \leq n(x,y)$.
With \autoref{thm:small_lcs_construction} (\ref{thm:small_lcs_construction:size}) and the definition of $A$ we can further bound $n(x,y) \leq \bigO{AD} = \bigO{L} \leq \bigO{m} \leq \bigO{n}$ where we again used $L \leq m \leq n$.
We further have $|\Sigma(x,y)| = 2 \leq \bigO{n^{\alpha_\Sigma}}$.
We can bound the number of matching pairs by $M(x,y) \leq n(x,y)^2 \leq \bigO{n^{\alpha_M}}$ where we used that $\alpha_M = 2$ holds in the current case.

In total we have shown that $p(x,y) \leq \bigO{p}$ for all $p \in P'$.
Hence, we have shown a conditional lower bound for every instance in $\lcsy{\mathbb{\alpha}}$ where $\mathbb{\alpha}$ satisfies \autoref{tab:restrictions}, $\alpha_\delta = \alpha_m$ and $\alpha_M = 2$.
In the next section, we now show how to lift the restriction $\alpha_M = 2$.

\subsubsection{Matching Pairs Reduction}

Before we lift the assumption $\alpha_M = 2$, we present why it was necessary in the previous reduction.
Let $(x,y)$ be the constructed \lcs{} instance as before.
Per definition of matching pairs, we know $M(x,y) \geq \numSymbols{0}{x} \cdot \numSymbols{0}{y}$.
Looking at the construction, we have $\bigO{A}$ blocks of $O^\gamma$ in $x$.
From \autoref{thm:small_lcs_construction} and \ref{thm:nvg} we further know $\gamma \geq |\nvg{a_i}| = \Theta(D)$. 
Hence, $\numSymbols{0}{x} \geq \bigO{AD}$.
In $y$ we have one block $0^{A\gamma'}$ at the beginning of $y$, so $\numSymbols{0}{y} \geq A\gamma' \geq \bigO{AD}$ where we bounded $\gamma'$ similar to $\gamma$.
In total we have $M(x,y) \geq \bigO{(AD)^2} = \bigO{n(x,y)^2}$ and therefore $\alpha_M \geq 2$.
However, there is a parameter setting with $\alpha_M < 2$, e.g., for $\Sigma = \{ \sigma_1, \ldots, \sigma_n \}$ we can define $x = y := \sigma_1 \cdots \sigma_n$.
In this case, we have $\alpha_m = \alpha_L = \alpha_\Sigma = \alpha_d = 1$, $\alpha_\delta = \alpha_\Delta = 0$ and $\alpha_M = 1 < 2$.


To reduce our matching pairs, we can look at the parameter relation from \autoref{sec:fin_param_rels}, specifically \autoref{thm:matching_pairs_lb}.
Here we have shown that $M \geq \frac{L^2}{|\Sigma|}$.
Hence, we have the chance to reduce $M$ by increasing the alphabet size.
Therefore, the authors split the input vector sets into multiple partitions.
They then used the previous reduction on each partition.
They combine these constructed strings while shifting them to different alphabets.
This is done in such a way that correctness remains while the used alphabet size increases, hence $M$ decreases for the constructed instances.
For the full construction and proof we refer to the original paper \cite[Section 9.1.3]{Bringman.2018}.

