\subsubsection{Large LCS}

We now assume that $\alpha_L = \alpha_m$, i.e., $L = \Theta(m)$.
This means that our constructed instances should match almost the full smaller string $y$ in an \lcs{}.
In the previous construction there were large unmatched parts, hence, this construction will not work in this case.
However, we will still use the \emph{normalized vector gadgets} from the previous construction, but we will change how we are embedding them into a full string.
Therefore, the authors created a so called \emph{1vs1/2vs1 gadget} \cite[section 9.2.1]{Bringman.2018}.
We will present this gadget in the following.

\paragraph{1vs1/2vs1 gadget}
The idea is to have two sets of strings $x_1, x_2, \ldots, x_P$ and $y_1, y_2, \ldots, y_Q$ embedded into $x$ and $y$.
This is done in such a way, that in an \lcs{} each gadget for $y_j$ is either matched with one or two gadgets from $x$.
In the former case the \lcs{} will only depend on the \lcs{} of the underlying string pairs $x_i$ and $y_j$ and in the latter case, the full gadget for $y_j$ will be matched.
See \autoref{fig:12vs1gadget} for a visualization of this.
To formalize the idea, the authors presented a lemma \cite[Lemma 9.6]{Bringman.2018} similar to the following.

\newcommand{\lLCSguard}[5]{%
	\node[base, cell, number, guard, minimum width=#4, #5] (#1) at (#2) {#3};
}
\newcommand{\lLCSinput}[5]{%
	\node[base, cell, number, input, minimum width=#4, #5] (#1) at (#2) {#3};
}
\newcommand{\lLCSguardedString}[4]{%
	\lLCSguard{#1_g_0}{base}{$0^{\gamma_1}$}{1.5em}{#4}
	\lLCSguard{#1_g_1_left}{#1_g_0.east}{$1^{\gamma_2}$}{1.5em}{#4}
	\lLCSguard{#1_g_01}{#1_g_1_left.east}{$(01)^{\gamma_3}$}{3em}{#4}
	\lLCSinput{#1}{#1_g_01.east}{#2}{#3}{#4}
	\lLCSguard{#1_g_1_right}{#1.east}{$1^{\gamma_3}$}{1.5em}{#4}
	\coordinate (next) at (#1_g_1_right.east);
}
\newcommand{\lLCSguardedStringTopLabel}[4]{%
	\lLCSguardedString{#1}{$#2$}{#3}{#4}

	% Define the lower-left and lower-right corners
    \coordinate (botLeft) at (#1_g_0.south west);  
    \coordinate (midLeft) at (#1_g_0.north west);  
    \coordinate (botRight) at (#1_g_1_right.south east);
    \coordinate (midRight) at (#1_g_1_right.north east);  
	
	%\node[draw, inner sep=0, minimum height=14pt, anchor=south west, text width=\labelLength] (#1_label) at (southWest) {$G(#2)$};
	\path[draw, #4, fill=none] (botLeft) -- (midLeft) -- ++(0,2.2ex) -| (botRight);
	\node[part label, anchor=south, #4] (#1_label) at ($(midLeft)!0.5!(midRight)$) {$G(#2)$};
}
\newcommand{\lLCSguardedStringBotLabel}[4]{%
	\lLCSguardedString{#1}{$#2$}{#3}{#4}

	% Define the lower-left and lower-right corners
    \coordinate (topLeft) at (#1_g_0.north west); 
    \coordinate (midLeft) at (#1_g_0.south west);   
    \coordinate (topRight) at (#1_g_1_right.north east);
    \coordinate (midRight) at (#1_g_1_right.south east);  
	
	%\node[draw, inner sep=0, minimum height=14pt, anchor=south west, text width=\labelLength] (#1_label) at (southWest) {$G(#2)$};
	\path[draw, #4, fill=none] (topLeft) -- (midLeft) -- ++(0,-2.2ex) -| (topRight);
	\node[part label, anchor=north, #4] (#1_label) at ($(midLeft)!0.5!(midRight)$) {$G(#2)$};
}

\newcommand{\uniqueAlignedMatches}[2]{%
	\path[guard edge] (#1_g_0.north) edge (#2_g_0.south);
	\path[guard edge] (#1_g_1_left.north) edge (#2_g_1_left.south);
	\path[guard edge] (#1_g_01.north) edge (#2_g_01.south);
	\path[guard edge] (#1_g_1_right.north) edge (#2_g_1_right.south);
	
	\path[input edge] (#1.north) edge (#2.south);
	
	\path[input edge, ultra thick, draw] (#1_g_0.south west) -- (#1_g_0.north west) -- (#2_g_0.south west) -- (#2_g_0.north west);
	\path[input edge, ultra thick, draw] (#1_g_1_right.south east) -- (#1_g_1_right.north east) -- (#2_g_1_right.south east) -- (#2_g_1_right.north east);
}

\newcommand{\doubleAlignedMatches}[3]{%
	\path[guard edge] (#1_g_0.north) edge (#2_g_0.south);
	\path[guard edge] (#1_g_1_left.north) edge (#2_g_1_left.south);
	\path[guard edge] (#1_g_01.north) edge (#2_g_01.south);

	\path[input edge] (#1.north) edge (#3_g_01.south);
	
	\path[guard edge] (#1_g_1_right.north) edge (#3_g_1_right.south);
	
}


\begin{figure}
	\centering
	\color{black}
	\begin{tikzpicture}[x=1.5em]
		\tikzstyle{base} = [minimum width=1.5em, inner sep=1pt, anchor=west]
		\tikzstyle{text element} = []
		\tikzstyle{number} = [minimum height=14pt]  
		\tikzstyle{guard} = [fill=gray]  
		\tikzstyle{guard edge} = [color=gray]
		\tikzstyle{big guard} = [fill=gray!75!black]
		\tikzstyle{big guard edge} = [color=gray!75!black]
		\tikzstyle{input} = [fill=cyan]
		\tikzstyle{input edge} = [color=cyan, thick]
		\tikzstyle{cell} = [draw, minimum width=1.5em, fill=white]
		\tikzstyle{part label} = [font=\tiny]
		\tikzstyle{hide} = [draw=white!75!black, text=white!75!black]
		
		\node[base, text element] (eqX) at (0,0) {$=$};
		\node[base, text element, anchor=east] (x) at (eqX.west) {$x$};
		
		\coordinate (base) at ($(eqX.east) + (1,0)$);
		\lLCSguardedStringTopLabel{x1}{x_1}{1.5em}{}
		\coordinate (base) at (next);
		\lLCSguardedStringTopLabel{x2}{x_2}{1.5em}{}
		\coordinate (base) at (next);
		\lLCSguardedStringTopLabel{x3}{x_3}{1.5em}{}
		\coordinate (base) at (next);
		\node[base, minimum width=3em] (x_dots) at (base) {$\cdots$};
		\coordinate (base) at (x_dots.east);
		\lLCSguardedStringTopLabel{xP}{x_P}{1.5em}{}
		
		%\path[draw] (x1_g_1_right.north east) -- (x1_g_1_right.south east);
		%\path[draw] (x2_g_1_right.north east) -- (x2_g_1_right.south east);
		%\path[draw, ultra thick] (x3_g_1_right.north east) -- (x1_g_1_right.south east);
		
		
		
		\node[base, text element] (eqY) at (0,-1.4) {$=$};
		\node[base, text element, anchor=east] (y) at (eqY.west) {$y$};
		
		\coordinate (base) at ($(eqY.east) + (1,0)$);
		\lLCSguardedStringBotLabel{y1}{y_1}{1.5em}{}
		\coordinate (base) at (next);
		\lLCSguardedStringBotLabel{y2}{y_2}{1.5em}{}
		\coordinate (base) at (next);
		\node[base, minimum width=3em] (y_dots) at (base) {$\cdots$};
		\coordinate (base) at (y_dots.east);
		\lLCSguardedStringBotLabel{yQ}{y_Q}{1.5em}{}
		
		\uniqueAlignedMatches{y1}{x1}
		\doubleAlignedMatches{y2}{x2}{x3}
		\uniqueAlignedMatches{yQ}{xP}
	
	\end{tikzpicture}
	\label{fig:1-2vs1gadget}
	\caption{Todo}
\end{figure}






\begin{lemma}
Let $\rho_0 > \rho_1$ and $P, Q \geq 1$ with $P = 2Q - N$ where $0 \leq N \leq Q$.
For any strings $x_1, x_2, \ldots, x_P$ of length $\ell_x$ and $y_1, y_2, \ldots, y_Q$ of length $\ell_y$ with $L(x_i, y_j) \in \{\rho_0, \rho_1\}$ for all pairs of strings $x_i$ and $y_j$, construct
\begin{align*}
	x &:= \lLCSGadget{x_1, \ldots, x_P} := \gLargeLCS{x_1}\gLargeLCS{x_2} \cdots \gLargeLCS{x_P}\\
	y &:= \lLCSGadget{y_1, \ldots, y_Q} := \gLargeLCS{y_1}\gLargeLCS{y_2} \cdots \gLargeLCS{y_Q}
\end{align*}
where $\gLargeLCS{w} = 0^{\gamma_1}1^{\gamma_2}(01)^{\gamma_3}w1^{\gamma_3}$ with $\gamma_3 = \ell_x + \ell_y$, $\gamma_2 = 8\gamma_3$ and $\gamma_1 = 6\gamma_2$.
We define $\Gamma := \gamma_1 + \gamma_2 + 3\gamma_3$.
It holds:
%
\begin{enumerate}[(i)]
\item\label{lem:1-2vs1gadget-1} $L(x,y) \geq Q\Gamma + (Q-N)\ell_y + N\rho_1$
\item\label{lem:1-2vs1gadget-2} $L(x,y) \geq Q\Gamma + (Q-N)\ell_y + (N-1)\rho_1 + \rho_0 \Leftrightarrow \exists j \in \indexSet{Q - N},1 \leq \lambda \leq \min\{N, j\}: L(x_{2j-\lambda}, y_j) = \rho_0$
\end{enumerate}
%
if and only if there are $j \in \indexSet{Q}$ and $j - Q + N \leq \lambda \leq \min\{N, j\}$ such that $L(x_{2j-\lambda}, y_j) = \rho_0$.

\end{lemma}

\begin{proof}
We first show that for $N=0$, i.e., $P=2Q$, it holds that $L(x,y) = |y| = Q\Gamma + Q\ell_y$.
Observe therefore, that $L(\gLargeLCS{x_i}\gLargeLCS{x_{i+1}}, \gLargeLCS{y_j}) = |\gLargeLCS{y_j}|$ as $0^{\gamma_1}1^{\gamma_2}(01)^{\gamma_3}$ is a prefix of $\gLargeLCS{x_i}$, $y_i$ is a substring of $(01)^{\gamma_3}$ ($\gamma_3 \geq \ell_y$) and $1^{\gamma_3}$ is a suffix of $\gLargeLCS{x_{i+1}}$. See the orange part in \autoref{fig:12vs1gadget} for a visualization of this.
As $P = 2Q$, we can hence split $x$ and $y$ in $Q$ parts such that
\[
	L(x,y) \geq \sum_{1\leq j \leq Q} L(\gLargeLCS{x_{2j-1}}\gLargeLCS{x_{2j}}, \gLargeLCS{y_j}) = \sum_{1\leq j \leq Q} |\gLargeLCS{y_j}| = |y| = Q\Gamma + Q\ell_y.
\]
Equality follows as $|y|$ is the maximal \lcs{} size.

We now show that $L(x,y) \geq Q\Gamma + N\rho_1$ for $N=Q$, i.e., $P=2Q - N = Q$.
Note that $L(\gLargeLCS{x_i}, \gLargeLCS{y_j}) = \Gamma + L(x_i, y_j) \geq \Gamma + \rho_1$ per definition of $G$.
The claim now follows by again splitting $x$ and $y$ in $Q$ parts such that
\[
	L(x,y) \geq \sum_{1\leq i \leq Q} L(\gLargeLCS{x_i}, \gLargeLCS{y_i}) \geq \sum_{1 \leq i \leq Q} \Gamma + \rho_1 = Q\Gamma + Q\rho_1 = Q\Gamma + N\rho_1.
\]


% =================== (i) =================================
We now prove (\ref{lem:1-2vs1gadget-1}).
Let therefore $N$ be arbitrary.
The idea is now to split $x$ and $y$ into two parts: one where every $y$-gadget can match with two $x$-gadgets and one where there is a 1-to-1 mapping.
We can then use both above claims to follow (\ref{lem:1-2vs1gadget-1}).
Therefore, we define $Q_1 := Q-N$, $P_1 := P-N = 2(Q-N) = 2Q_1$.
Because $\mathcal{G}$ is linear, we can split $x$ and $y$ into the two parts
\begin{align*}
x &= \lLCSGadget{x_1, \ldots, x_{P_1}}\lLCSGadget{x_{P_1+1}, \ldots, x_{P}} =: x_{2vs1}x_{1vs1} \\
%
y &= \lLCSGadget{y_1, \ldots, y_{Q_1}}\lLCSGadget{y_{Q_1+1}, \ldots, y_{Q}} =: y_{2vs1}y_{1vs1}.
\end{align*}
%
Now we can use the claims before as $P_1 = 2Q_1$ and $P - P_1 = Q - Q_1 = N$ to limit 
\begin{align*}
L(x_{2vs1}, y_{2vs1}) &\geq Q_1\Gamma + Q_1\ell_y %
& \text{and}\\
%
L(x_{1vs1}, y_{1vs1}) &\geq Q_2\Gamma + Q_2\rho_1 = Q_2\Gamma + N\rho_1, 
\end{align*}
with which we can obtain the desired bound
\begin{align*}
L(x,y) &\geq L(\lLCSGadget{x^1_1, \ldots, x^1_{P_1}}, \lLCSGadget{y^1_1, \ldots, y^1_{Q_1}}) + L(\lLCSGadget{x^2_1, \ldots, x^2_{P_2}}, \lLCSGadget{y^2_1, \ldots, y^2_{Q_2}}) \\
	&\geq Q_1\Gamma + Q_1\ell_y + Q_2\Gamma + N\rho_1 = Q\Gamma + (Q-N)\ell_y + N\rho_1.
\end{align*}


% =================== (ii) =================================
We finalize our proof by showing (\ref{lem:1-2vs1gadget-2}).
Therefore, first assume there are $j \in \indexSet{Q}$ and $1 \leq \lambda \leq \min\{N, j\}$ such that $L(x_{2j-\lambda}, y_j) = \rho_0$.
Let $i := 2j - \lambda$.
We now split the input strings $x$ and $y$ as follows:
\begin{align*}
x &= \lLCSGadget{x_1, \ldots, x_{i-1}} \cdot \gLargeLCS{x_i} \cdot \lLCSGadget{x_{i+1}, \ldots, x_P} %
=: \lLCSGadget{\hat{x}_1, \ldots, \hat{x}_{P'}} \cdot \gLargeLCS{x_i} \cdot \lLCSGadget{\tilde{x}_{1}, \ldots, \tilde{x}_{P''}}\\
y &= \lLCSGadget{y_1, \ldots, y_{j-1}} \cdot \gLargeLCS{y_j} \cdot \lLCSGadget{y_{j+1}, \ldots, y_Q} %
=: \lLCSGadget{\hat{y}_1, \ldots, \hat{y}_{Q'}} \cdot \gLargeLCS{y_j} \cdot \lLCSGadget{\tilde{y}_{1}, \ldots, \tilde{y}_{Q''}}
\end{align*}
We want to apply (\ref{lem:1-2vs1gadget-1}) to the first and third part of this separation.
To do this we must first show that there is a $0 \leq N' \leq Q'$ such that $P' = 2Q' - N'$ and similarly an $N''$ for $P''$ and $Q''$.
Per definition $Q' = j-1$ and hence for $P'$ it follows that 
\[
P' = 2j - \lambda - 1 = 2(j-1) - (\lambda - 1) = 2Q' - N'
\] 
for $N' = \lambda - 1$. Because $\lambda \leq j$ it follows that $N' \leq j - 1 = Q'$.

We further have $Q'' = Q - j$ and 
\[
P'' = P - i = 2Q - N - 2j + \lambda = 2(Q-j) - (N - \lambda) = 2Q'' - N''
\]
for $N'' := N - \lambda$.
It remains to show that $0 \leq N'' \leq Q''$.
Because $1 \leq \lambda \leq N$ the first inequality is clear.
Further, $N - \lambda \leq N \leq Q - j \leq Q''$ where the second inequality is due to $j \leq Q - N$.






% ==============================================================
We now assume that there is no $j$ and $\lambda$ such that $L(x_{2j-\lambda}, y_j) = \rho_0$.


\end{proof}

