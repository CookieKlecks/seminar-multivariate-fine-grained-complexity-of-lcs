\subsection{Large LCS}

We now assume that $\alpha_L = \alpha_m$, i.e., $L = \Theta(m)$.
This means that our constructed instances should match almost the full smaller string $y$ in an \lcs{}.
In the previous construction there were large unmatched parts, hence, that construction would not work in this case.
However, we will still use the \emph{normalized vector gadgets} from the previous construction, but we will change how we are embedding them into a full string.
Therefore, the authors created a so called \emph{1vs1/2vs1 gadget} \cite[section 9.2.1]{Bringman.2018}.
We will present this gadget in the following.

\subsubsection{1vs1/2vs1 gadget}
The idea is to have two sets of strings $x_1, x_2, \ldots, x_P$ and $y_1, y_2, \ldots, y_Q$ embedded into $x$ and $y$.
This is done in such a way, that in an \lcs{} each gadget for $y_j$ is either matched with one or two gadgets from $x$.
In the former case the \lcs{} will only depend on the \lcs{} of the underlying string pairs $x_i$ and $y_j$ and in the latter case, the full gadget for $y_j$ will be matched.
See \autoref{fig:12vs1gadget} for a visualization of this.
To formalize the idea, the authors presented a lemma \cite[Lemma 9.6]{Bringman.2018} similar to the following. \todo{Maybe split lemma in def and several lemma}

\newcommand{\lLCSguard}[5]{%
	\node[base, cell, number, guard, minimum width=#4, #5] (#1) at (#2) {#3};
}
\newcommand{\lLCSinput}[5]{%
	\node[base, cell, number, input, minimum width=#4, #5] (#1) at (#2) {#3};
}
\newcommand{\lLCSguardedString}[4]{%
	\lLCSguard{#1_g_0}{base}{$0^{\gamma_1}$}{1.5em}{#4}
	\lLCSguard{#1_g_1_left}{#1_g_0.east}{$1^{\gamma_2}$}{1.5em}{#4}
	\lLCSguard{#1_g_01}{#1_g_1_left.east}{$(01)^{\gamma_3}$}{3em}{#4}
	\lLCSinput{#1}{#1_g_01.east}{#2}{#3}{#4}
	\lLCSguard{#1_g_1_right}{#1.east}{$1^{\gamma_3}$}{1.5em}{#4}
	\coordinate (next) at (#1_g_1_right.east);
}
\newcommand{\lLCSguardedStringTopLabel}[5]{%
	\lLCSguardedString{#1}{$#2$}{#3}{#4}

	% Define the lower-left and lower-right corners
    \coordinate (botLeft) at (#1_g_0.south west);  
    \coordinate (midLeft) at (#1_g_0.north west);  
    \coordinate (botRight) at (#1_g_1_right.south east);
    \coordinate (midRight) at (#1_g_1_right.north east);  
	
	%\node[draw, inner sep=0, minimum height=14pt, anchor=south west, text width=\labelLength] (#1_label) at (southWest) {$G(#2)$};
	\path[draw, #5] (botLeft) -- (midLeft) -- ++(0,2.2ex) -| (botRight);
	\node[part label, anchor=south, #4] (#1_label) at ($(midLeft)!0.5!(midRight)$) {$G(#2)$};
}
\newcommand{\lLCSguardedStringBotLabel}[5]{%
	\lLCSguardedString{#1}{$#2$}{#3}{#4}

	% Define the lower-left and lower-right corners
    \coordinate (topLeft) at (#1_g_0.north west); 
    \coordinate (midLeft) at (#1_g_0.south west);   
    \coordinate (topRight) at (#1_g_1_right.north east);
    \coordinate (midRight) at (#1_g_1_right.south east);  
	
	%\node[draw, inner sep=0, minimum height=14pt, anchor=south west, text width=\labelLength] (#1_label) at (southWest) {$G(#2)$};
	\path[draw, #5] (topLeft) -- (midLeft) -- ++(0,-2.2ex) -| (topRight);
	\node[part label, anchor=north, #4] (#1_label) at ($(midLeft)!0.5!(midRight)$) {$G(#2)$};
}

\newcommand{\uniqueAlignedMatches}[2]{%
	\path[guard edge] (#1_g_0.north) edge (#2_g_0.south);
	\path[guard edge] (#1_g_1_left.north) edge (#2_g_1_left.south);
	\path[guard edge] (#1_g_01.north) edge (#2_g_01.south);
	\path[guard edge] (#1_g_1_right.north) edge (#2_g_1_right.south);
	
	\path[input edge] (#1.north) edge (#2.south);
	
	\path[unique align, draw] (#1_g_0.south west) -- (#1_g_0.north west) -- (#2_g_0.south west) -- (#2_g_0.north west);
	\path[unique align, draw] (#1_g_1_right.south east) -- (#1_g_1_right.north east) -- (#2_g_1_right.south east) -- (#2_g_1_right.north east);
}

\newcommand{\doubleAlignedMatches}[3]{%
	\path[guard edge] (#1_g_0.north) edge (#2_g_0.south);
	\path[guard edge] (#1_g_1_left.north) edge (#2_g_1_left.south);
	\path[guard edge] (#1_g_01.north) edge (#2_g_01.south);

	\path[input edge] (#1.north) edge (#3_g_01.south);
	
	\path[guard edge] (#1_g_1_right.north) edge (#3_g_1_right.south);
	
	
	\path[double align, draw] (#1_g_0.south west) -- (#1_g_0.north west) -- (#2_g_0.south west) -- (#2_g_0.north west);
	\path[double align, draw] (#1_g_1_right.south east) -- (#1_g_1_right.north east) -- (#3_g_1_right.south east) -- (#3_g_1_right.north east);
}


\begin{figure}
	\centering
	\color{black}
	\begin{tikzpicture}[x=1.5em]
		\tikzstyle{base} = [minimum width=1.5em, inner sep=1pt, anchor=west]
		\tikzstyle{text element} = []
		\tikzstyle{number} = [minimum height=14pt]  
		\tikzstyle{guard} = [fill=lightgray]  
		\tikzstyle{guard edge} = [color=gray]
		\tikzstyle{big guard} = [fill=gray!75!black]
		\tikzstyle{big guard edge} = [color=gray!75!black]
		\tikzstyle{input} = [fill=cyan]
		\tikzstyle{input edge} = [color=cyan, thick]
		\tikzstyle{cell} = [draw, minimum width=1.5em, fill=white]
		\tikzstyle{part label} = [font=\tiny]
		\tikzstyle{hide} = [draw=white!75!black, text=white!75!black]
		\tikzstyle{unique align} = [color=teal, ultra thick]
		\tikzstyle{double align} = [color=orange, ultra thick]		
		
		\node[base, text element] (eqX) at (0,0) {$=$};
		\node[base, text element, anchor=east] (x) at (eqX.west) {$x$};
		
		\coordinate (base) at ($(eqX.east) + (1,0)$);
		\lLCSguardedStringTopLabel{x1}{x_1}{1.5em}{}{unique align}
		\coordinate (base) at ($(next) + (0.1,0)$);
		\lLCSguardedStringTopLabel{x2}{x_2}{1.5em}{}{double align}
		\coordinate (base) at (next);
		\lLCSguardedStringTopLabel{x3}{x_3}{1.5em}{}{double align}
		\coordinate (base) at (next);
		\node[base, minimum width=3em] (x_dots) at (base) {$\cdots$};
		\coordinate (base) at (x_dots.east);
		\lLCSguardedStringTopLabel{xP}{x_P}{1.5em}{}{unique align}
		
		%\path[draw] (x1_g_1_right.north east) -- (x1_g_1_right.south east);
		%\path[draw] (x2_g_1_right.north east) -- (x2_g_1_right.south east);
		%\path[draw, ultra thick] (x3_g_1_right.north east) -- (x1_g_1_right.south east);
		
		
		
		\node[base, text element] (eqY) at (0,-1.4) {$=$};
		\node[base, text element, anchor=east] (y) at (eqY.west) {$y$};
		
		\coordinate (base) at ($(eqY.east) + (1,0)$);
		\lLCSguardedStringBotLabel{y1}{y_1}{1.5em}{}{unique align}
		\coordinate (base) at ($(next) + (0.1,0)$);
		\lLCSguardedStringBotLabel{y2}{y_2}{1.5em}{}{double align}
		\coordinate (base) at (next);
		\node[base, minimum width=3em] (y_dots) at (base) {$\cdots$};
		\coordinate (base) at (y_dots.east);
		\lLCSguardedStringBotLabel{yQ}{y_Q}{1.5em}{}{unique align}
		
		\uniqueAlignedMatches{y1}{x1}
		\doubleAlignedMatches{y2}{x2}{x3}
		\uniqueAlignedMatches{yQ}{xP}
	
	\end{tikzpicture}
	\caption{Visualization of the 1vs1/2vs1 gadget. Possible mappings of $y$ to $x$ are visualized. Here $G(y_1)$ is uniquely mapped to $G(x_1)$ and $G(y_2)$ to both $G(x_2)$ and $G(x_3)$. Note that in the first case the \lcs{} only depends on $L(x_1, y_1)$ as the other parts can be fully matched. In the latter case, the full string $G(y_2)$ is a substring of $G(x_2)G(x_3)$ by matching $y_2$ with the $(01)^{\gamma_3}$ part.}
	\label{fig:12vs1gadget}
\end{figure}






\begin{lemma}
\label{lem:1-2vs1gadget}
Let $\rho_0 > \rho_1$ and $P, Q \geq 1$ with $P = 2Q - N$.
For any strings $x_1, x_2, \ldots, x_P$ of length $\ell_x$ and $y_1, y_2, \ldots, y_Q$ of length $\ell_y$ with $L(x_i, y_j) \in \{\rho_0, \rho_1\}$ for all pairs of strings $x_i$ and $y_j$, construct
\begin{align*}
	x &:= \lLCSGadget{x_1, \ldots, x_P} := \gLargeLCS{x_1}\gLargeLCS{x_2} \cdots \gLargeLCS{x_P}\\
	y &:= \lLCSGadget{y_1, \ldots, y_Q} := \gLargeLCS{y_1}\gLargeLCS{y_2} \cdots \gLargeLCS{y_Q}
\end{align*}
where $\gLargeLCS{w} = 0^{\gamma_1}1^{\gamma_2}(01)^{\gamma_3}w1^{\gamma_3}$ with $\gamma_3 = \ell_x + \ell_y$, $\gamma_2 = 8\gamma_3$ and $\gamma_1 = 6\gamma_2$.
We define $\Gamma := \gamma_1 + \gamma_2 + 3\gamma_3$.
It holds:
%
\begin{enumerate}[(i)]
\item\label{lem:1-2vs1gadget:not-ortho} if $L(x_i, y_j) = \rho_1$ for all $i \in \indexSet{P}, j \in \indexSet{Q}$: 
\[ 
L(x,y) \leq Q\Gamma + (Q-N)\ell_y + N\rho_1
\]
\item\label{lem:1-2vs1gadget:ortho} if there is $j \in \indexSet{Q}, \lambda \in \indexSet{N}$ s.t. $L(x_{2j-\lambda}, y_j) = \rho_0$: 
\[
L(x,y) \geq Q\Gamma + (Q-N)\ell_y + (N-1)\rho_1 + \rho_0
\]


%\item\label{lem:1-2vs1gadget-1} $L(x,y) \geq Q\Gamma + (Q-N)\ell_y + N\rho_1$
%\item\label{lem:1-2vs1gadget-2} $L(x,y) \geq Q\Gamma + (Q-N)\ell_y + (N-1)\rho_1 + \rho_0 \Leftrightarrow \exists j \in \indexSet{Q - N},1 \leq \lambda \leq \min\{N, j\}: L(x_{2j-\lambda}, y_j) = \rho_0$
\end{enumerate}
%
%if and only if there are $j \in \indexSet{Q}$ and $j - Q + N \leq \lambda \leq \min\{N, j\}$ such that $L(x_{2j-\lambda}, y_j) = \rho_0$.

\end{lemma}

\begin{proof}
%We first show that for $N=0$, i.e., $P=2Q$, it holds that $L(x,y) = |y| = Q\Gamma + Q\ell_y$.
%Observe therefore, that $L(\gLargeLCS{x_i}\gLargeLCS{x_{i+1}}, \gLargeLCS{y_j}) = |\gLargeLCS{y_j}|$ as $0^{\gamma_1}1^{\gamma_2}(01)^{\gamma_3}$ is a prefix of $\gLargeLCS{x_i}$, $y_i$ is a substring of $(01)^{\gamma_3}$ ($\gamma_3 \geq \ell_y$) and $1^{\gamma_3}$ is a suffix of $\gLargeLCS{x_{i+1}}$. See the orange part in \autoref{fig:12vs1gadget} for a visualization of this.
%As $P = 2Q$, we can hence split $x$ and $y$ in $Q$ parts such that
%\[
%	L(x,y) \geq \sum_{1\leq j \leq Q} L(\gLargeLCS{x_{2j-1}}\gLargeLCS{x_{2j}}, \gLargeLCS{y_j}) = \sum_{1\leq j \leq Q} |\gLargeLCS{y_j}| = |y| = Q\Gamma + Q\ell_y.
%\]
%Equality follows as $|y|$ is the maximal \lcs{} size.
%
%We now show that $L(x,y) \geq Q\Gamma + N\rho_1$ for $N=Q$, i.e., $P=2Q - N = Q$.
%Note that $L(\gLargeLCS{x_i}, \gLargeLCS{y_j}) = \Gamma + L(x_i, y_j) \geq \Gamma + \rho_1$ per definition of $G$.
%The claim now follows by again splitting $x$ and $y$ in $Q$ parts such that
%\[
%	L(x,y) \geq \sum_{1\leq i \leq Q} L(\gLargeLCS{x_i}, \gLargeLCS{y_i}) \geq \sum_{1 \leq i \leq Q} \Gamma + \rho_1 = Q\Gamma + Q\rho_1 = Q\Gamma + N\rho_1.
%\]



% ====================== Proof non-ortho =======================
To prove (\ref{lem:1-2vs1gadget:not-ortho}), we define $x =: z_1z_2 \cdots z_Q$ such that $L(x,y) = \sum_{j=1}^Q L(z_j, \gLargeLCS{y_j}$.
We say that $x_i$ is aligned with $y_j$ if and only if $z_j$ contains strictly more than half of the prefix $0^{\gamma_1}$ of $\gLargeLCS{x_i}$.
Note that every $x_i$ is assigned to at most one $y_j$.
We say that $y_j$ is $k$-aligned if $k$ different $x_{i_1}, \ldots, x_{i_k}$ are aligned with $y_j$.
For the following, we define $H(w) := 1^{\gamma_2}(01)^{\gamma_3}w1^{\gamma_3}$

First consider any $0$-aligned $y_j$.
Per definition is $z_j$ a subsequence of $0^{\gamma_1 / 2}H(x_i)0^{\gamma / 2}$.
By using greedy prefix matching (\autoref{lem:greedy_prefix_match}), we can bound
\[
	L(z_j,\gLargeLCS{y_j}) \leq L(0^{\gamma_1 / 2}H(x_i)0^{\gamma / 2}, 0^{\gamma_1}H(y_j)) \leq \frac{\gamma_1}{2} + L(H(x_i)0^{\gamma / 2}, 0^{\gamma_1 / 2}H(y_j))
\]
By \autoref{lem:guards} with $\ell := \gamma_{1}/2 \geq 2\gamma_{2} + 6\gamma_{3} + \ell_{X} + \ell_{Y}
= |H(x_{i})| + |H(y_{j})| \geq \numSymbols{0}{H(x_{i})} + |H(y_{j})|$, 
\[
L\left(H(x_{i})0^{\gamma_{1}/2}, 0^{\gamma_{1}/2}H(y_{j})\right)
= \frac{\gamma_{1}}{2} + L\left(0^{\numSymbols{0}{H(x_{i})}}, H(y_{j})\right)
\leq \frac{\gamma_{1}}{2} + \numSymbols{0}{H(y_{j}} 
\leq \frac{\gamma_{1}}{2} + \gamma_{3} + \ell_{Y}.
\]
Together we have $L(z_j,\gLargeLCS{y_j}) \leq \gamma_1 + \gamma_3 + \ell_y =: L_0$.

With a similar analysis we can show, that when $y_j$ is $1$-aligned with $x_i$ we have
\[
	L(z_j, \gLargeLCS{y_j}) \leq \Gamma + L(x_i, y_j) = \gamma_1 + \gamma_2 + 3\gamma_3 + \rho_1 =: L_1
\]
where we used the assumption that $L(x_{i'}, y_{j'}) = \rho_1$ for every $i', j'$.

In the case that $y_j$ is $k$-aligned for $k \geq 2$, we can trivially bound $L(z_j, \gLargeLCS{y_j}) \leq |\gLargeLCS{y_j}| = \Gamma + \ell_y =: L_k$.

Let now $n_i$ be the number of $i$-aligned $y_j$.
We set $n_k := \sum_{2\leq i} n_i$.
Because ever $y_j$ has exactly one alignment category, we have $n_0 + n_1 + n_k = Q$.
Further $2n_k + n_1 \leq P$ as every $k$-aligned $y_j$ uses $k$ unique $x_i$s.
By the previous bounds, we can bound
\[
	L(x,y) = \sum_{j=1}^Q L(z_j, \gLargeLCS{y_j} \leq n_0 L_0 + n_1 L_1 + n_k L_k
\]
To find the maximum of this bound, note first $L_0 < L_1 \leq L_k$.
Hence, we should maximize $n_k$.
However, note also that $L_0 + L_k = 2\gamma_1 + \gamma_2 + 4\gamma_3 + 2\ell_y < 2\Gamma \leq 2L_1$.
This shows that it is better to maximize $n_1$ such that $n_0 = 0$ instead of increasing both $n_0$ and $n_k$ by one.
In total, we maximize the above bound by setting $n_k = Q-N$ and $n_1 = N$ which gives us the desired bound. \todo{improve maximize argument}


% ========================== Claim =============================
Before proving (\ref{lem:1-2vs1gadget:ortho}), we first show the following claim.
\begin{claim}
\label{claim:1-2vs1-gadget-lb}
We have $L(x,y) \geq Q\Gamma + (Q-N)\ell_y + N\rho_1$
\end{claim} 
Let therefore $N$ be arbitrary.
Observe that for $N=0$, i.e., $P=2Q$, it holds that $L(x,y) = |y| = Q\Gamma + Q\ell_y$ and $L(x,y) \geq Q\Gamma + \sum_i L(x_i, y_i) \geq Q\Gamma + Q\rho_1$ for $N=Q$, i.e., $P = Q$.
The idea is now to split $x$ and $y$ into two parts: one where every $y$-gadget can match with two $x$-gadgets and one where there is a 1-to-1 mapping.
We can then use both observations to follow the claim.
Therefore, we define $Q_1 := Q-N$, $P_1 := P-N = 2(Q-N) = 2Q_1$.
Because of how $\mathcal{G}$ is defined, we can split $x$ and $y$ into the two parts
\begin{align*}
x &= \lLCSGadget{x_1, \ldots, x_{P_1}}\lLCSGadget{x_{P_1+1}, \ldots, x_{P}} =: x_{2vs1}x_{1vs1} \\
%
y &= \lLCSGadget{y_1, \ldots, y_{Q_1}}\lLCSGadget{y_{Q_1+1}, \ldots, y_{Q}} =: y_{2vs1}y_{1vs1}.
\end{align*}
%
Now we can use the observations as $P_1 = 2Q_1$ and $Q_2 := Q - Q_1 = N = P - P_1$ to limit 
\begin{align*}
L(x_{2vs1}, y_{2vs1}) &\geq Q_1\Gamma + Q_1\ell_y %
& \text{and}\\
%
L(x_{1vs1}, y_{1vs1}) &\geq Q_2\Gamma + Q_2\rho_1 = Q_2\Gamma + N\rho_1, 
\end{align*}
with which we can obtain the desired bound
\begin{align*}
L(x,y) &\geq L(x_{2vs1}, y_{2vs1}) + L(x_{1vs1}, y_{1vs1}) \\
	&\geq Q_1\Gamma + Q_1\ell_y + Q_2\Gamma + N\rho_1 = Q\Gamma + (Q-N)\ell_y + N\rho_1.
\end{align*}



% ============================ ortho ===========================
Let us now prove (\ref{lem:1-2vs1gadget:ortho}).
Therefore, first assume there are $j \in \indexSet{Q}$ and $\lambda \in \indexSet{N}$ such that $L(x_{2j-\lambda}, y_j) = \rho_0$.
Let $i := 2j - \lambda$.
We now split the input strings $x$ and $y$ as follows:
\begin{align*}
x &= \lLCSGadget{x_1, \ldots, x_{i-1}} \cdot \gLargeLCS{x_i} \cdot \lLCSGadget{x_{i+1}, \ldots, x_P} %
=: \lLCSGadget{\hat{x}_1, \ldots, \hat{x}_{P'}} \cdot \gLargeLCS{x_i} \cdot \lLCSGadget{\tilde{x}_{1}, \ldots, \tilde{x}_{P''}}\\
y &= \lLCSGadget{y_1, \ldots, y_{j-1}} \cdot \gLargeLCS{y_j} \cdot \lLCSGadget{y_{j+1}, \ldots, y_Q} %
=: \lLCSGadget{\hat{y}_1, \ldots, \hat{y}_{Q'}} \cdot \gLargeLCS{y_j} \cdot \lLCSGadget{\tilde{y}_{1}, \ldots, \tilde{y}_{Q''}}
\end{align*}
We want to apply \autoref{claim:1-2vs1-gadget-lb} to the first and third part of this separation.
To do this we must first show that there is a $N'$ such that $P' = 2Q' - N'$ and similarly an $N''$ for $P''$ and $Q''$.
Per definition $Q' = j-1$ and hence for $P'$ it follows that 
\[
P' = i - 1 = 2j - \lambda - 1 = 2(j-1) - (\lambda - 1) = 2Q' - N'
\] 
for $N' = \lambda - 1$. Note that $\lambda \geq 1$.
%
We further have $Q'' = Q - j$ and 
\[
P'' = P - i = 2Q - N - 2j + \lambda = 2(Q-j) - (N - \lambda) = 2Q'' - N''
\]
for $N'' := N - \lambda$ which is non-negative as $\lambda \leq N$.
Hence, \autoref{claim:1-2vs1-gadget-lb} is applicable and we can bound
\begin{align*}
	L(x,y) &\geq L\left( \lLCSGadget{\hat{x}_1, \ldots, \hat{x}_{P'}}, \lLCSGadget{\hat{y}_1, \ldots, \hat{y}_{Q'}} \right) + L(\gLargeLCS{x_i}, \gLargeLCS{y_j}) + L\left(\lLCSGadget{\tilde{x}_{1}, \ldots, \tilde{x}_{P''}}, \lLCSGadget{\tilde{y}_{1}, \ldots, \tilde{y}_{Q''}} \right)\\
	%
	&\geq Q'\Gamma + (Q' - N')\ell_y + N'\rho_1  + Q''\Gamma + (Q'' - N'')\ell_y + N''\rho_1 + L(\gLargeLCS{x_i}, \gLargeLCS{y_j})\\
	&\geq (Q-1)\Gamma + (Q-N)\ell_y + (N-1)\rho_1 + L(\gLargeLCS{x_i}, \gLargeLCS{y_j})
\end{align*}
as $Q' + Q'' = Q-1$, $N' + N'' = N - 1$ and $(Q' - N') + (Q'' - N'') = (Q-1) - (N -1) = Q-N$.
With the bound $L(\gLargeLCS{x_i}, \gLargeLCS{y_j}) \geq \Gamma + L(x_i, y_j)$ and the assumption that $L(x_i, y_j) = \rho_0$ it follows (\ref{lem:1-2vs1gadget:ortho}).
%
%
%
%
%
%
%
%
%
% =================== (i) =================================
%We now prove (\ref{lem:1-2vs1gadget-1}).
%Let therefore $N$ be arbitrary.
%The idea is now to split $x$ and $y$ into two parts: one where every $y$-gadget can match with two $x$-gadgets and one where there is a 1-to-1 mapping.
%We can then use both above claims to follow (\ref{lem:1-2vs1gadget-1}).
%Therefore, we define $Q_1 := Q-N$, $P_1 := P-N = 2(Q-N) = 2Q_1$.
%Because $\mathcal{G}$ is linear, we can split $x$ and $y$ into the two parts
%\begin{align*}
%x &= \lLCSGadget{x_1, \ldots, x_{P_1}}\lLCSGadget{x_{P_1+1}, \ldots, x_{P}} =: x_{2vs1}x_{1vs1} \\
%%
%y &= \lLCSGadget{y_1, \ldots, y_{Q_1}}\lLCSGadget{y_{Q_1+1}, \ldots, y_{Q}} =: y_{2vs1}y_{1vs1}.
%\end{align*}
%%
%Now we can use the claims before as $P_1 = 2Q_1$ and $P - P_1 = Q - Q_1 = N$ to limit 
%\begin{align*}
%L(x_{2vs1}, y_{2vs1}) &\geq Q_1\Gamma + Q_1\ell_y %
%& \text{and}\\
%%
%L(x_{1vs1}, y_{1vs1}) &\geq Q_2\Gamma + Q_2\rho_1 = Q_2\Gamma + N\rho_1, 
%\end{align*}
%with which we can obtain the desired bound
%\begin{align*}
%L(x,y) &\geq L(\lLCSGadget{x^1_1, \ldots, x^1_{P_1}}, \lLCSGadget{y^1_1, \ldots, y^1_{Q_1}}) + L(\lLCSGadget{x^2_1, \ldots, x^2_{P_2}}, \lLCSGadget{y^2_1, \ldots, y^2_{Q_2}}) \\
%	&\geq Q_1\Gamma + Q_1\ell_y + Q_2\Gamma + N\rho_1 = Q\Gamma + (Q-N)\ell_y + N\rho_1.
%\end{align*}


% =================== (ii) =================================
%We finalize our proof by showing (\ref{lem:1-2vs1gadget-2}).
%Therefore, first assume there are $j \in \indexSet{Q}$ and $1 \leq \lambda \leq \min\{N, j\}$ such that $L(x_{2j-\lambda}, y_j) = \rho_0$.
%Let $i := 2j - \lambda$.
%We now split the input strings $x$ and $y$ as follows:
%\begin{align*}
%x &= \lLCSGadget{x_1, \ldots, x_{i-1}} \cdot \gLargeLCS{x_i} \cdot \lLCSGadget{x_{i+1}, \ldots, x_P} %
%=: \lLCSGadget{\hat{x}_1, \ldots, \hat{x}_{P'}} \cdot \gLargeLCS{x_i} \cdot \lLCSGadget{\tilde{x}_{1}, \ldots, \tilde{x}_{P''}}\\
%y &= \lLCSGadget{y_1, \ldots, y_{j-1}} \cdot \gLargeLCS{y_j} \cdot \lLCSGadget{y_{j+1}, \ldots, y_Q} %
%=: \lLCSGadget{\hat{y}_1, \ldots, \hat{y}_{Q'}} \cdot \gLargeLCS{y_j} \cdot \lLCSGadget{\tilde{y}_{1}, \ldots, \tilde{y}_{Q''}}
%\end{align*}
%We want to apply (\ref{lem:1-2vs1gadget-1}) to the first and third part of this separation.
%To do this we must first show that there is a $0 \leq N' \leq Q'$ such that $P' = 2Q' - N'$ and similarly an $N''$ for $P''$ and $Q''$.
%Per definition $Q' = j-1$ and hence for $P'$ it follows that 
%\[
%P' = 2j - \lambda - 1 = 2(j-1) - (\lambda - 1) = 2Q' - N'
%\] 
%for $N' = \lambda - 1$. Because $\lambda \leq j$ it follows that $N' \leq j - 1 = Q'$.
%
%We further have $Q'' = Q - j$ and 
%\[
%P'' = P - i = 2Q - N - 2j + \lambda = 2(Q-j) - (N - \lambda) = 2Q'' - N''
%\]
%for $N'' := N - \lambda$.
%It remains to show that $0 \leq N'' \leq Q''$.
%Because $1 \leq \lambda \leq N$ the first inequality is clear.
%Further, $N - \lambda \leq N \leq Q - j \leq Q''$ where the second inequality is due to $j \leq Q - N$.
%
%
%
%
%
%
%% ==============================================================
%We now assume that there is no $j$ and $\lambda$ such that $L(x_{2j-\lambda}, y_j) = \rho_0$.
%
\end{proof}

We now use this lemma to define our construction for reducing \ov{} instances to \lcs{}.

\begin{lemma}
\label{lem:large_lcs_reduce}
For an \ov{} instance $\mathcal{A} = \{a_1, \ldots, a_A\}$ and $\mathcal{B} = \{b_1, \ldots, b_B\}$ with $A \mid B$, we define strings $x_i := \nvg{a_i}$ for $i \in \indexSet{A}$ and $y_j := \nvg{b_j}$ for $j \in \indexSet{B}$ with lengths $\ell_x, \ell_y = \bigO{D}$ and construct
\begin{align*}
\tilde{x} &:= (\tilde{x}_1, \ldots, \tilde{x}_P) 
= \overbrace{\left(x_1, \ldots, x_A, x_1, \ldots, x_A, \ldots, x_1, \ldots, x_A\right)}^{\text{$2\cdot(B/A)+3$ groups of size $A$}} 
, \\
\tilde{y} &:= (\tilde{y}_1, \ldots, \tilde{y}_Q) 
= \underbrace{(y_1, \ldots, y_1}_{\text{$A$ copies of $y_1$}},
y_1, \ldots, y_B, 
\underbrace{y_1, \ldots, y_1)}_{\text{$A$ copies of $y_1$}}
,
\end{align*}
where $P := 2B + 3A$ and $Q := B + 2A$, i.e., $P = 2Q - A$. The strings $x := \lLCSGadget{\tilde{x_1}, \ldots, \tilde{x_P}}$ and $y := \lLCSGadget{\tilde{y_1}, \ldots, \tilde{y_P}}$ satisfy the following:

\begin{enumerate}[(i)]
\item\label{lem:large_lcs_reduce:infer}
$L(x,y) \geq Q\Gamma + (Q-A)\ell_y + (A-1)\rho_1 + \rho_0 =: \rho$ if and only if there are $i \in \indexSet{A}, j \in \indexSet{B}$ with $\langle a_i, b_j \rangle = 0$.
%
\item\label{lem:large_lcs_reduce:size}
It holds that $|y|, |x| = \bigO{BD}$ and $\delta(x,y) = \bigO{AD}$
\end{enumerate}
\end{lemma}

\begin{proof}
To show (\ref{lem:large_lcs_reduce:infer}) we first assume that no $i \in \indexSet{A}, j \in \indexSet{B}$ with $\langle a_i, b_j \rangle = 0$ exist.
By definition of \nvgName{} we have $L(x_i, y_j) = L(\nvg{a_i}, \nvg{b_j}) = \rho_1$ for all $i \in \indexSet{A}, j \in \indexSet{B}$.
With $\rho_0 > \rho_1$ and \autoref{lem:1-2vs1gadget} (\ref{lem:1-2vs1gadget:not-ortho}) it follows that $L(x,y) < \rho$.

Now assume there is $i \in \indexSet{A}, j \in \indexSet{B}$ with $\langle a_i, b_j \rangle = 0$.
Note that there is a $j' \in \indexSet{Q}$ with $j' \geq A$ s.t. $\tilde{y}_{j'} = y_j$.
It remains to show that there is a $\lambda \in \indexSet{A}$ with $\tilde{x}_{2j - \lambda} = x_i$.
To show this, note that $(\tilde{x}_{2j - 1}, \tilde{x}_{2j - 2}, \ldots, \tilde{x}_{2j - A})$ is a consecutive sequence of $A$ strings in $\tilde{x}$.
Due to the cyclic structure of $\tilde{x}$ we know that this sequence includes every $x_i$ for $i \in \indexSet{A}$.
This concludes the proof for (\ref{lem:large_lcs_reduce:infer}) with \autoref{lem:1-2vs1gadget} (\ref{lem:1-2vs1gadget:ortho}).

To prove (\ref{lem:large_lcs_reduce:size}) first observe that $Q \leq P \leq \bigO{B}$ as $A \leq B$. With $\ell_x, \ell_y \leq \bigO{D}$ we obtain $|x|,|y| \leq \bigO{BD}$.
For the bound of $\delta(x,y)$ we can use \autoref{claim:1-2vs1-gadget-lb} for
\[
L(x,y) \geq Q\Gamma + (Q - A)\ell_y \geq Q\Gamma + Q\ell_y - \bigO{A\ell_y} = |y| - \bigO{AD}
\]
which implies $\delta(x,y) = |y| - L(x,y) \leq \bigO{AD}$
\end{proof}



\paragraph*{Reduction}
We now use \autoref{lem:large_lcs_reduce} to instantiate the actual reductions with the desired bounds.
Let $n \geq 1$ be arbitrary and consider any parameter setting $\alpha$ with $\alpha_L = \alpha_m$.
We write $\lfloor x \rfloor_2$ for the largest power of $2$ less or equal to $x$.
We construct strings $x$ and $y$ as in \autoref{lem:large_lcs_reduce} with $D = n^{o(1)}$, $A := \lfloor \frac{\min\{\delta, d\}}{mD} \rfloor_2$ and $B := \lfloor \frac{m}{D} \rfloor_2$.\footnote{In the original paper they used slightly different values for $A$ and $B$. We however simplified this, because we later restrict us to a reduced parameter space for simplicity reasons.}
Note that \uovh{} implies a lower bound of $(AB)^{1-o(1)} \leq n^{\alpha_d - o(1)} = d^{1 - o(1)}$ which is one of the lower bounds of the main result.\todo{mention lemma with main result}
To use \autoref{lem:large_lcs_reduce} we have to show $A \leq B$ which implies $A \mid B$.
This follows from $\delta \leq m$.
Hence, we can infer from $L(x,y)$ whether any vectors are orthogonal by \autoref{lem:large_lcs_reduce} (\ref{lem:large_lcs_reduce:infer}).
Further the running time of the reduction is $\bigO{BD} = \bigO{m} = \bigO{L} \leq \bigO{d}$ as our current case assumes $\alpha_m = \alpha_L$ and $L \leq d$ (see \autoref{tab:restrictions}).
This proves a lower bound for any algorithm for $\lcsy{\mathbb{\alpha}}$ of $d^{1-o(1)}$.


What now remains to be shown is the parameter space that this reduction spans, i.e., for what \lcs{} parameter combinations we can proof a conditional lower bound.







\paragraph*{Parameter Space}
Recall that in the current case we assume $\alpha_L = \alpha_m$ and we write $p = n^{\alpha_p}$ as the target value for any parameter $p \in \mathcal{P}$ and $p(x,y)$ as the actual value for the \lcs{} instance $(x,y)$.
Further our goal is again to show that given a target parameter setting $\mathbf{\alpha}$, we can create instances that are in $\lcsy{\mathbf{\alpha}}$.
We will now show this using a reduced parameter set of $P' = \{n, m, L, \delta\}$ for the current case.
For a proof of the full parameter set we refer to the original paper \cite[section 9.2.2]{Bringman.2018}.
We have trivially and by definition $L(x,y) \leq m(x,y) \leq n(x,y)$.
With \autoref{lem:large_lcs_reduce} (\ref{lem:large_lcs_reduce:size}) and the definition of $B$ we can further bound $n(x,y) \leq \bigO{BD} = \bigO{m} \leq \bigO{n}$ where we again used $m \leq n$.
Using \autoref{lem:large_lcs_reduce} (\ref{lem:large_lcs_reduce:size}) we can bound $\delta(x,y) \leq \bigO{AD} \leq \bigO{\delta}$.

In total we have shown that $p(x,y) \leq \bigO{p}$ for all $p \in P'$.
Hence, we have shown a conditional lower bound for every instance in $\lcsy{\mathbb{\alpha}}$ where $\mathbb{\alpha}$ satisfies \autoref{tab:restrictions} and $\alpha_L = \alpha_m$.
%
Note that we ignored here the parameter $M$.
If we would have considered matching pairs we would have a similar problem as in the case of $\alpha_\delta = \alpha_m$.
Therefore, the authors provided again an adaption that increases the used alphabet size (see \cite[section 9.2.3]{Bringman.2018}).