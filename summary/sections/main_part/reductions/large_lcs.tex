\subsection{Large LCS}

We now assume that $\alpha_L = \alpha_m$, i.e., our constructed instances should match almost the full string $y$.
The previous construction had large unmatched parts, proving it unsuitable here.
However, we will still use the \nvgName{}s as basis, but change how we are embedding them into a full string.
Therefore, the authors created a so called \emph{1vs1/2vs1 gadget} \cite[section 9.2.1]{Bringman.2018}.
%We will present this gadget in the following.

\subsubsection{1vs1/2vs1 gadget}
The idea is to embed the strings $x_1, x_2, \ldots, x_P$ and $y_1, y_2, \ldots, y_Q$ into $x$ and $y$, s.t. in an \lcs{} each gadget for $y_j$ is either matched with one or two from $x$.
In the former case the \lcs{} will only depend on the \lcs{} of the underlying $x_i$ and $y_j$ and in the latter case, the full gadget for $y_j$ will be matched.
See \autoref{fig:12vs1gadget} for a visualization of this.

\newcommand{\lLCSguard}[5]{%
	\node[base, cell, number, guard, minimum width=#4, #5] (#1) at (#2) {#3};
}
\newcommand{\lLCSinput}[5]{%
	\node[base, cell, number, input, minimum width=#4, #5] (#1) at (#2) {#3};
}
\newcommand{\lLCSguardedString}[4]{%
	\lLCSguard{#1_g_0}{base}{$0^{\gamma_1}$}{1.5em}{#4}
	\lLCSguard{#1_g_1_left}{#1_g_0.east}{$1^{\gamma_2}$}{1.5em}{#4}
	\lLCSguard{#1_g_01}{#1_g_1_left.east}{$(01)^{\gamma_3}$}{3em}{#4}
	\lLCSinput{#1}{#1_g_01.east}{#2}{#3}{#4}
	\lLCSguard{#1_g_1_right}{#1.east}{$1^{\gamma_3}$}{1.5em}{#4}
	\coordinate (next) at (#1_g_1_right.east);
}
\newcommand{\lLCSguardedStringTopLabel}[5]{%
	\lLCSguardedString{#1}{$#2$}{#3}{#4}

	% Define the lower-left and lower-right corners
    \coordinate (botLeft) at (#1_g_0.south west);  
    \coordinate (midLeft) at (#1_g_0.north west);  
    \coordinate (botRight) at (#1_g_1_right.south east);
    \coordinate (midRight) at (#1_g_1_right.north east);  
	
	%\node[draw, inner sep=0, minimum height=14pt, anchor=south west, text width=\labelLength] (#1_label) at (southWest) {$G(#2)$};
	\path[draw, #5] (botLeft) -- (midLeft) -- ++(0,2.2ex) -| (botRight);
	\node[part label, anchor=south, #4] (#1_label) at ($(midLeft)!0.5!(midRight)$) {$G(#2)$};
}
\newcommand{\lLCSguardedStringBotLabel}[5]{%
	\lLCSguardedString{#1}{$#2$}{#3}{#4}

	% Define the lower-left and lower-right corners
    \coordinate (topLeft) at (#1_g_0.north west); 
    \coordinate (midLeft) at (#1_g_0.south west);   
    \coordinate (topRight) at (#1_g_1_right.north east);
    \coordinate (midRight) at (#1_g_1_right.south east);  
	
	%\node[draw, inner sep=0, minimum height=14pt, anchor=south west, text width=\labelLength] (#1_label) at (southWest) {$G(#2)$};
	\path[draw, #5] (topLeft) -- (midLeft) -- ++(0,-2.2ex) -| (topRight);
	\node[part label, anchor=north, #4] (#1_label) at ($(midLeft)!0.5!(midRight)$) {$G(#2)$};
}

\newcommand{\uniqueAlignedMatches}[2]{%
	\path[guard edge] (#1_g_0.north) edge (#2_g_0.south);
	\path[guard edge] (#1_g_1_left.north) edge (#2_g_1_left.south);
	\path[guard edge] (#1_g_01.north) edge (#2_g_01.south);
	\path[guard edge] (#1_g_1_right.north) edge (#2_g_1_right.south);
	
	\path[input edge] (#1.north) edge (#2.south);
	
	\path[unique align, draw] (#1_g_0.south west) -- (#1_g_0.north west) -- (#2_g_0.south west) -- (#2_g_0.north west);
	\path[unique align, draw] (#1_g_1_right.south east) -- (#1_g_1_right.north east) -- (#2_g_1_right.south east) -- (#2_g_1_right.north east);
}

\newcommand{\doubleAlignedMatches}[3]{%
	\path[guard edge] (#1_g_0.north) edge (#2_g_0.south);
	\path[guard edge] (#1_g_1_left.north) edge (#2_g_1_left.south);
	\path[guard edge] (#1_g_01.north) edge (#2_g_01.south);

	\path[input edge] (#1.north) edge (#3_g_01.south);
	
	\path[guard edge] (#1_g_1_right.north) edge (#3_g_1_right.south);
	
	
	\path[double align, draw] (#1_g_0.south west) -- (#1_g_0.north west) -- (#2_g_0.south west) -- (#2_g_0.north west);
	\path[double align, draw] (#1_g_1_right.south east) -- (#1_g_1_right.north east) -- (#3_g_1_right.south east) -- (#3_g_1_right.north east);
}


\begin{figure}
	\centering
	\color{black}
	\begin{tikzpicture}[x=1.5em]
		\tikzstyle{base} = [minimum width=1.5em, inner sep=1pt, anchor=west]
		\tikzstyle{text element} = []
		\tikzstyle{number} = [minimum height=14pt]  
		\tikzstyle{guard} = [fill=lightgray]  
		\tikzstyle{guard edge} = [color=gray]
		\tikzstyle{big guard} = [fill=gray!75!black]
		\tikzstyle{big guard edge} = [color=gray!75!black]
		\tikzstyle{input} = [fill=cyan]
		\tikzstyle{input edge} = [color=cyan, thick]
		\tikzstyle{cell} = [draw, minimum width=1.5em, fill=white]
		\tikzstyle{part label} = [font=\tiny]
		\tikzstyle{hide} = [draw=white!75!black, text=white!75!black]
		\tikzstyle{unique align} = [color=teal, ultra thick]
		\tikzstyle{double align} = [color=orange, ultra thick]		
		
		\node[base, text element] (eqX) at (0,0) {$=$};
		\node[base, text element, anchor=east] (x) at (eqX.west) {$x$};
		
		\coordinate (base) at ($(eqX.east) + (1,0)$);
		\lLCSguardedStringTopLabel{x1}{x_1}{1.5em}{}{unique align}
		\coordinate (base) at ($(next) + (0.1,0)$);
		\lLCSguardedStringTopLabel{x2}{x_2}{1.5em}{}{double align}
		\coordinate (base) at (next);
		\lLCSguardedStringTopLabel{x3}{x_3}{1.5em}{}{double align}
		\coordinate (base) at (next);
		\node[base, minimum width=3em] (x_dots) at (base) {$\cdots$};
		\coordinate (base) at (x_dots.east);
		\lLCSguardedStringTopLabel{xP}{x_P}{1.5em}{}{unique align}
		
		%\path[draw] (x1_g_1_right.north east) -- (x1_g_1_right.south east);
		%\path[draw] (x2_g_1_right.north east) -- (x2_g_1_right.south east);
		%\path[draw, ultra thick] (x3_g_1_right.north east) -- (x1_g_1_right.south east);
		
		
		
		\node[base, text element] (eqY) at (0,-1.4) {$=$};
		\node[base, text element, anchor=east] (y) at (eqY.west) {$y$};
		
		\coordinate (base) at ($(eqY.east) + (1,0)$);
		\lLCSguardedStringBotLabel{y1}{y_1}{1.5em}{}{unique align}
		\coordinate (base) at ($(next) + (0.1,0)$);
		\lLCSguardedStringBotLabel{y2}{y_2}{1.5em}{}{double align}
		\coordinate (base) at (next);
		\node[base, minimum width=3em] (y_dots) at (base) {$\cdots$};
		\coordinate (base) at (y_dots.east);
		\lLCSguardedStringBotLabel{yQ}{y_Q}{1.5em}{}{unique align}
		
		\uniqueAlignedMatches{y1}{x1}
		\doubleAlignedMatches{y2}{x2}{x3}
		\uniqueAlignedMatches{yQ}{xP}
	
	\end{tikzpicture}
	\caption{Visualization of the 1vs1/2vs1 gadget. Possible mappings of $y$ to $x$ are visualized. Here $G(y_1)$ is uniquely mapped to $G(x_1)$ and $G(y_2)$ to both $G(x_2)$ and $G(x_3)$. Note that in the first case the \lcs{} only depends on $L(x_1, y_1)$ as the other parts can be fully matched. In the latter case, the full string $G(y_2)$ is a substring of $G(x_2)G(x_3)$ by matching $y_2$ with the $(01)^{\gamma_3}$ part.}
	\label{fig:12vs1gadget}
\end{figure}



\begin{definition}[1vs1/2vs1 Gadget]
Fix $P, Q \geq 1$ with $P = 2Q - N$.
For $\rho_0 > \rho_1 \geq 0$ and strings $x_1, \ldots, x_P$ and $y_1, \ldots, y_Q$ of length $\ell_x$ and $\ell_y$ with $L(x_i, y_j) \in \{\rho_0, \rho_1\}$, we define
\begin{align*}
	\lLCSGadget{x_1, \ldots, x_P} &:= \gLargeLCS{x_1}\gLargeLCS{x_2} \cdots \gLargeLCS{x_P}\\
	\lLCSGadget{y_1, \ldots, y_Q} &:= \gLargeLCS{y_1}\gLargeLCS{y_2} \cdots \gLargeLCS{y_Q}
\end{align*}
where $\gLargeLCS{w} = 0^{\gamma_1}1^{\gamma_2}(01)^{\gamma_3}w1^{\gamma_3}$ with $\gamma_3 = \ell_x + \ell_y$, $\gamma_2 = 8\gamma_3$ and $\gamma_1 = 6\gamma_2$.
We further define $\Gamma := \gamma_1 + \gamma_2 + 3\gamma_3$.
We say that $\lLCSGadgetName$ is the 1vs1/2vs1 gadget of these strings.
\end{definition}

We will now prove some properties of this gadget.
As notation we use in the following $x := \lLCSGadget{x_1, \ldots, x_P}$ and $y := \lLCSGadget{y_1, \ldots, y_Q}$ for any appropriate $P$, $Q$ and strings $x_i$ and $y_j$.
We show an upper bound for the case that $L(x_i, y_i) = \rho_1$ for all $x_i$ and $y_j$ and a lower bound when one $L(x_i, y_j) = \rho_0$ for appropriate $x_i$ and $y_j$ exists.
This will later translate to the cases where no orthogonal vectors or one $\langle a_i, b_j \rangle = 0$ exists respectively.


\begin{lemma}
\label{lem:1-2vs1-gadget-edge-cases}
For $N = 0$ and $N = Q$, i.e., $P = 2Q$ and $P = Q$ respectively, we have
\begin{enumerate}[(i)]
\item\label{lem:1-2vs1-gadget-edge-cases:P=2Q} if $P = 2Q$: $L(\lLCSGadget{x_1, \ldots, x_P}, \lLCSGadget{y_1, \ldots, y_Q}) \geq Q(\Gamma + \ell_y)$
\item\label{lem:1-2vs1-gadget-edge-cases:P=Q} if $P = Q$: $L(\lLCSGadget{x_1, \ldots, x_P}, \lLCSGadget{y_1, \ldots, y_Q}) \geq Q\Gamma + \sum_{i = 1}^{Q} L(x_i, y_i)$
\end{enumerate}
\end{lemma}
\begin{proof}
For (\ref{lem:1-2vs1-gadget-edge-cases:P=2Q}) observe that $L(\gLargeLCS{u}\gLargeLCS{u'}, \gLargeLCS{v}) \geq |\gLargeLCS{w}|$.
Hence, if we split the strings such that we always match two $x$-gadgets with one $y$-gadget, we can match the full $y$ string and $|\lLCSGadget{y_1, \ldots, y_Q}| = Q(\Gamma + \ell_y)$.
Analogous for (\ref{lem:1-2vs1-gadget-edge-cases:P=Q}) with $L(\gLargeLCS{u}, \gLargeLCS{v}) \geq \Gamma + L(u,v)$.
This is because we can greedily match all guarding blocks.
Both cases are visualized in \autoref{fig:12vs1gadget}.
\end{proof}



Now we can start with the case if one $L(x_i, y_j) = \rho_0$, i.e., the vectors are orthogonal.

\begin{lemma}
\label{lem:1-2vs1:ortho-lower-bound}
If there are $N < j \leq Q-N$ and $\lambda \in \indexSet{N}$ s.t. $L(x_{2j-\lambda}, y_j) = \rho_0$, we have
\[ 
L(x,y) \geq Q\Gamma + (Q-N)\ell_y + (N-1)\rho_1 + \rho_0.
\]
We define $i := 2j - \lambda$, i.e., $L(x_i, y_j) = \rho_0$.
\end{lemma}

\begin{proof}
Assume such $j$ and $\lambda$ exist.
We split the strings $x$ and $y$ into the three parts
\begin{align*}
x = \lLCSGadget{x_1, \ldots, x_P} &= \lLCSGadget{x^L_1, \ldots, x^L_{P_L}} \lLCSGadget{x^M_1, \ldots, x^M_N} \lLCSGadget{x^R_1, \ldots, x^R_{P_R}} =: \lLCSGadgetName^L_x\lLCSGadgetName^M_x\lLCSGadgetName^R_x \quad\text{and}\\
y = \lLCSGadget{y_1, \ldots, y_Q} &= \lLCSGadget{y^L_1, \ldots, y^L_{Q_L}} \lLCSGadget{y^M_1, \ldots, y^M_N} \lLCSGadget{y^R_1, \ldots, y^R_{Q_R}} =: \lLCSGadgetName^L_y\lLCSGadgetName^M_y\lLCSGadgetName^R_y,
\end{align*}
where $Q_L = j - \lambda$, $P_L = 2(j - \lambda)$, $Q_R = Q - Q_L - N$ and $P_R = P - P_L - N$.
We now want to lower bound the three parts. %$L^L := L(\lLCSGadgetName^L_x, \lLCSGadgetName^L_y)$, $L^M := L(\lLCSGadgetName^M_x, \lLCSGadgetName^M_y)$ and $L^R := L(\lLCSGadgetName^R_x, \lLCSGadgetName^R_y)$.
For the left and right part, note that $P_L = 2Q_L$ and $P_R = 2Q_R$, hence we can apply \autoref{lem:1-2vs1-gadget-edge-cases} (\ref{lem:1-2vs1-gadget-edge-cases:P=2Q}) to obtain
\begin{align*}
L(\lLCSGadgetName^L_x, \lLCSGadgetName^L_y) &\geq Q_L(\Gamma + \ell_y) \\
L(\lLCSGadgetName^R_x, \lLCSGadgetName^R_y) &\geq Q_R(\Gamma + \ell_y) = (Q - Q_L - N)(\Gamma + \ell_y).
\end{align*}
For the middle part, we first show that $x_i$ and $y_j$ are actual in the middle part, i.e., $x_i = x^M_{i'}$ and $y_j = y^M_{j'}$ for some $i', j' \in \indexSet{N}$.
Note therefore that
\begin{align*}
P_L = 2j - 2\lambda &<& &2j - \lambda = i& &\leq 2j - \lambda - \lambda + N = P_L + N \quad\text{and}\\
Q_L = j - \lambda &<& &j& &\leq j - \lambda + N = Q_L + N,
\end{align*}
where we used $1 \leq \lambda \leq N$.
Now we can apply \autoref{lem:1-2vs1-gadget-edge-cases} (\ref{lem:1-2vs1-gadget-edge-cases:P=Q}) with $P=Q=N$ and use that $L(x^M_{i'}, y^M_{j'}) = L(x_i, y_j) = \rho_0$ and $L(x^M_k, y^M_k) \geq \rho_1$ for all $k \in \indexSet{N}$ to bound
\[
L(\lLCSGadgetName^M_x, \lLCSGadgetName^M_y) \geq N\Gamma + \sum_{k=1}^N L(x^M_k, y^M_k) \geq N\Gamma + (N-1)\rho_1 + \rho_0.
\]
Finally, we combine the results and obtain the desired bound
\begin{align*}
L(x,y) &\geq L(\lLCSGadgetName^L_x, \lLCSGadgetName^L_y) + L(\lLCSGadgetName^M_x, \lLCSGadgetName^M_y) + L(\lLCSGadgetName^R_x, \lLCSGadgetName^R_y)\\
	&\geq Q_L(\Gamma + \ell_y) + N\Gamma + (N-1)\rho_1 + \rho_0 + (Q - Q_L - N)(\Gamma + \ell_y) \\
	&= Q\Gamma + (Q-N)\ell_y + (N-1)\rho_1 + \rho_0.
\end{align*}
We can now continue with the case that $L(x_i, y_j) = \rho_1$ for all $i, j$.
\end{proof}


\begin{lemma}
\label{lem:1-2vs1:non-ortho-upper-bound}
If $L(x_i, y_j) = \rho_1$ for all $i \in \indexSet{P}, j \in \indexSet{Q}$, we have
\[ 
L(x,y) \leq Q\Gamma + (Q-N)\ell_y + N\rho_1
\]
\end{lemma}


\begin{proof}
%
We define $x =: z_1z_2 \cdots z_Q$ such that $L(x,y) = \sum_{j=1}^Q L(z_j, \gLargeLCS{y_j})$.
We say that $x_i$ is aligned with $y_j$ if and only if $z_j$ contains strictly more than half of the prefix $0^{\gamma_1}$ of $\gLargeLCS{x_i}$.
Note that every $x_i$ is assigned to at most one $y_j$.
We say that $y_j$ is $k$-aligned if $k$ different $x_{i_1}, \ldots, x_{i_k}$ are aligned with $y_j$.
For the following, we define $H(w) := 1^{\gamma_2}(01)^{\gamma_3}w1^{\gamma_3}$.

First consider any $0$-aligned $y_j$.
Per definition is $z_j$ a subsequence of $0^{\frac{\gamma_1}{2}}H(x_i)0^{\frac{\gamma_1}{2}}$.
It is possible to show that we can bound
\[
L(z_j, \gLargeLCS{y_j}) = L(0^{\frac{\gamma_1}{2}}H(x_i)0^{\frac{\gamma_1}{2}}, 0^{\gamma_1}H(y_j)) \leq \gamma_1 + \gamma_3 + \ell_y =: L_0.
\]
For a full proof of this, we refer to the original paper \cite[Lemma 9.6\footnote{Lemma 9.6 combines this and the following lemma in this summary. The proof of the claim is in the proof for the second inequality of Lemma 9.6.}]{Bringman.2018}.
But to give an intuition observe, that $y_1$ is chosen so big, that it is always optimal to match the full $0^{\gamma_1}$ block.
This match also stretches over $H(x_i)$.
The remainder $H(y_j)$ then can only match with remaining zeroes in the right $0^{\frac{\gamma_1}{2}}$ block.
With $\numSymbols{0}{H(y_j)} \leq y_3 + \ell_y$ we obtain the bound.


Due to appropriate $\gamma_1$, $\gamma_2$ and $\gamma_3$, one can further show that when $y_j$ is $1$-aligned we have
\[
	L(z_j, \gLargeLCS{y_j}) \leq \Gamma + L(x_i, y_j) = \Gamma + \rho_1 =: L_1,
\]
where we used the assumption that $L(x_{i'}, y_{j'}) = \rho_1$ for every $i', j'$.
For a full proof we again refer to the original paper.
%
Further when $y_j$ is $k$-aligned for $k \geq 2$, we can trivially bound 
\[
L(z_j, \gLargeLCS{y_j}) \leq |\gLargeLCS{y_j}| = \Gamma + \ell_y =: L_k.
\]

Let now $n_0$, $n_1$ and $n_k$ be the number of $0$-, $1$- and $k$-aligned $y_j$ ($k \geq 2$).
Because ever $y_j$ has exactly one alignment category, we have $n_0 + n_1 + n_k = Q$.
Further $2n_k + n_1 \leq P$ as every $k$-aligned $y_j$ is aligned with $k$ unique $x_i$s and every $x_i$ with at most one $y_j$.
By the previous bounds, we obtain
\begin{equation}
\label{eq:1-2vs1:non-ortho-upper-bound:eq-ns}
	L(x,y) = \sum_{j=1}^Q L(z_j, \gLargeLCS{y_j} \leq n_0 L_0 + n_1 L_1 + n_k L_k 
\end{equation}
It remains to show that this is bounded by $Q\Gamma + (Q-N)\ell_y + N\rho_1$. 
For this we show the following claim.
\begin{claim}
Setting $n_0 = \hat{n}_0 := 0$, $n_1 = \hat{n}_1 := N$ and $n_k = \hat{n}_k := (Q-N)$ maximizes
\[
\mathcal{L}(n_0, n_1, n_k) := n_0 L_0 + n_1 L_1 + n_k L_k
\]
under the assumptions $n_0 + n_1 + n_k = Q$ and $2n_k + n_1 \leq P = 2Q - N$.
\end{claim}
Observe first that these specific values satisfy the assumptions.
Now assume $n_0 > 0$ would maximize $\mathcal{L}(n_0, n_1, n_k)$ for any $n_1$ and $n_k$.
If we define $n'_0 := n_0 - 1$, $n'_1 := n_1 + 2$ and $n'_k := n_k - 1$, we have
\[
	\mathcal{L}(n'_0, n'_1, n'_k) = \mathcal{L}(n_0, n_1, n_k) + 2L_1 - L_0 - L_k > \mathcal{L}(n_0, n_1, n_k),
\]
as $L_0 + L_k = \Gamma + \gamma_1 + \gamma_3 + 2\ell_y < \Gamma + \gamma_1 + \gamma_2 + \gamma_3 < 2\Gamma \leq 2L_1$.
That is a contradiction, hence, $n_0 = 0$ if we want to maximize $\mathcal{L}(n_0, n_1, n_k)$.
Now we obtain from the first assumption $n_1 \leq Q - n_k$.
Putting this into the second assumption, we obtain
\begin{align*}
2n_k + Q - n_k &\leq 2Q - N %
&\Leftrightarrow& %
&n_k \leq Q - N.
\end{align*}
Hence, $\mathcal{L}(n_0, n_1, n_k)$ is maximized for $n_k = Q-N$, because $L_k \geq L_1 \geq 0$. 
The claim follows.



Now we can further bound \autoref{eq:1-2vs1:non-ortho-upper-bound:eq-ns} by
\begin{align*}
L(x,y) \leq \hat{n}_0 L_0 + \hat{n}_1 L_1 + \hat{n}_k L_k &= NL_1 + (Q-N) L_k  \\
	&= N\Gamma + N\rho_1 + (Q-N)(\Gamma + \ell_y) \\
	&= Q\Gamma + (Q-N)\ell_y + N\rho_1
\end{align*}
\end{proof}










We now use these lemmas to define our construction for reducing \ov{} instances to \lcs{}.

\begin{lemma}
\label{lem:large_lcs_reduce}
For an \ov{} instance $\mathcal{A} = \{a_1, \ldots, a_A\}$ and $\mathcal{B} = \{b_1, \ldots, b_B\}$ with $A \mid B$, we define strings $x_i := \nvg{a_i}$  and $y_j := \nvg{b_j}$ for $i \in \indexSet{A}, j \in \indexSet{B}$ and construct
\begin{align*}
\tilde{x} &:= (\tilde{x}_1, \ldots, \tilde{x}_P) 
= \overbrace{\left(x_1, \ldots, x_A, x_1, \ldots, x_A, \ldots, x_1, \ldots, x_A\right)}^{\text{$2\cdot(B/A)+3$ groups of size $A$}} 
, \\
\tilde{y} &:= (\tilde{y}_1, \ldots, \tilde{y}_Q) 
= \underbrace{(y_1, \ldots, y_1}_{\text{$A$ copies of $y_1$}},
y_1, \ldots, y_B, 
\underbrace{y_1, \ldots, y_1)}_{\text{$A$ copies of $y_1$}}
,
\end{align*}
where $P := 2B + 3A$ and $Q := B + 2A$.
We define $\rho := Q\Gamma + (Q-A)\ell_y + (A-1)\rho_1 + \rho_0$.
The strings $x := \lLCSGadget{\tilde{x_1}, \ldots, \tilde{x_P}}$ and $y := \lLCSGadget{\tilde{y_1}, \ldots, \tilde{y_P}}$ satisfy:

\begin{enumerate}[(i)]
\item\label{lem:large_lcs_reduce:infer}
$L(x,y) \geq \rho$ if and only if there are $i \in \indexSet{A}, j \in \indexSet{B}$ with $\langle a_i, b_j \rangle = 0$.
%
\item\label{lem:large_lcs_reduce:size}
It holds that $|y|, |x| = \bigO{BD}$ and $\delta(x,y) = \bigO{AD}$
\end{enumerate}
\end{lemma}

\begin{proof}
To show (\ref{lem:large_lcs_reduce:infer}) we first assume that no $i \in \indexSet{A}, j \in \indexSet{B}$ with $\langle a_i, b_j \rangle = 0$ exist.
By \autoref{thm:nvg} we have $L(x_i, y_j) = L(\nvg{a_i}, \nvg{b_j}) = \rho_1$ for all $i \in \indexSet{A}, j \in \indexSet{B}$.
With $\rho_0 > \rho_1$ and \autoref{lem:1-2vs1:non-ortho-upper-bound} it follows that $L(x,y) < \rho$.

Now assume there are $i \in \indexSet{A}, j \in \indexSet{B}$ with $\langle a_i, b_j \rangle = 0$.
We show the requirements of \autoref{lem:1-2vs1:ortho-lower-bound}.
Per construction there is a $A < j' \leq Q - A$ such that $\tilde{y}_{j'} = y_j$.
We define $X := \{\tilde{x}_{2j' - \lambda} \mid \lambda \in \indexSet{A}\}$, which is a consecutive section of $\tilde{x}$.
Because of the cyclic structure of $\tilde{x}$ and $|X| = A$ we know that $x_i \in X$.
Hence, there is an $\lambda$ s.t. $L(\tilde{x}_{2j' - \lambda}, y_{j'}) = L(x_i, y_j) = \rho_0$.
With this we can apply \autoref{lem:1-2vs1:ortho-lower-bound} with $N = A$ to prove (\ref{lem:large_lcs_reduce:infer}).

For (\ref{lem:large_lcs_reduce:size}) first observe that $Q \leq P \leq \bigO{B}$ as $A \leq B$. With $\ell_x, \ell_y \leq \bigO{D}$ we obtain $|x|,|y| \leq \bigO{BD}$.
For $\delta(x,y)$ we can use \autoref{lem:1-2vs1-gadget-edge-cases} (\ref{lem:1-2vs1-gadget-edge-cases:P=2Q}) on a prefix of $x$ and $y$ to bound
\[
L(x,y) \geq (Q-A)(\Gamma + \ell_y) \geq Q(\Gamma + \ell_y) - \bigO{A\ell_y} = |y| - \bigO{AD}
\]
where we used $P - A = 2(Q-A)$. This implies $\delta(x,y) = |y| - L(x,y) \leq \bigO{AD}$
\end{proof}



\paragraph*{Reduction}
We now use \autoref{lem:large_lcs_reduce} to instantiate the actual reductions. %with the desired bounds.
Let $n \geq 1$ be arbitrary and consider any parameter setting $\alpha$ with $\alpha_L = \alpha_m$.
We write $\lfloor x \rfloor_2$ for the largest power of $2$ less or equal to $x$.
We construct strings $x$ and $y$ as in \autoref{lem:large_lcs_reduce} with $D = n^{o(1)}$, $A := \lfloor \frac{\min\{\delta, d\}}{mD} \rfloor_2$ and $B := \lfloor \frac{m}{D} \rfloor_2$.\footnote{In the original paper they used slightly different values for $A$ and $B$. We however simplified this, because we later restrict us to a reduced parameter space for simplicity reasons.}
Note that \uovh{} implies a lower bound of $(AB)^{1-o(1)} \leq n^{\alpha_d - o(1)} = d^{1 - o(1)}$.
To use \autoref{lem:large_lcs_reduce} we have to show $A \leq B$ which implies $A \mid B$.
This follows from $\delta \leq m$.
Hence, we can infer from $L(x,y)$ whether any vectors are orthogonal by \autoref{lem:large_lcs_reduce} (\ref{lem:large_lcs_reduce:infer}).
Further the running time of the reduction is $\bigO{BD} = \bigO{m} = \bigO{L} \leq \bigO{d}$ as we assume $\alpha_m = \alpha_L$ and $L \leq d$ (see \autoref{tab:restrictions}).
This proves a lower bound for $\lcsy{\mathbb{\alpha}}$ of $d^{1-o(1)}$.


%What now remains to be shown is the parameter space that this reduction spans, i.e., for what \lcs{} parameter combinations we can proof a conditional lower bound.







\paragraph*{Parameter Space}
Recall that we now have to show that given $\mathbf{\alpha}$ satisfying \autoref{tab:restrictions} and $\alpha_m = \alpha_L$, we can create instances in $\lcsy{\mathbf{\alpha}}$.
We again use a reduced parameter set of $P'' = \{n, m, L, \delta\}$.\footnote{For a proof of the full parameter set we refer to the original paper \cite[section 9.2.2]{Bringman.2018}.}
We again have $L(x,y) \leq m(x,y) \leq n(x,y)$.
With \autoref{lem:large_lcs_reduce} (\ref{lem:large_lcs_reduce:size}) and the definition of $B$ we can bound $n(x,y) \leq \bigO{BD} = \bigO{m} \leq \bigO{n}$ due to $m \leq n$.
Using \autoref{lem:large_lcs_reduce} (\ref{lem:large_lcs_reduce:size}) we can bound $\delta(x,y) \leq \bigO{AD} \leq \bigO{\delta}$.
%
In total we have shown that $p(x,y) \leq \bigO{p}$ for all $p \in P''$.
Hence, we have a conditional lower bound for every instance in $\lcsy{\mathbb{\alpha}}$. %where $\mathbb{\alpha}$ satisfies \autoref{tab:restrictions} and $\alpha_L = \alpha_m$.

Note that we ignored here the parameter $M$.
If we would have considered matching pairs we would have a similar problem as in the case of $\alpha_\delta = \alpha_m$.
The authors provided again an adaption that increases the used alphabet size for this case (see \cite[Section 9.2.3]{Bringman.2018}).