\section{Find Parameter Relations}
\label{sec:fin_param_rels}
The goal of this step is to find valid and non-trivial parameter combinations.
This is important as we want to restrict the parameter space that must be spanned by the reductions in the next step.

Before finding the actual relations, we first define a so called \emph{parameter setting} and its non-triviality.

\begin{definition}[Parameter Setting]
Let $\mathcal{P}$ be the set of parameters.
We define $\mathbb{\alpha} = (\alpha_p)_{p \in \mathcal{P}}$ as a \emph{parameter setting}.
We further define $\lcsy{\mathbb{\alpha}}$ as the subset of \lcs{} instances, where the input strings $x$ and $y$ fulfill
\begin{displaymath}
	\forall p \in \mathcal{P}: \  \frac{n^{\alpha_p}}{\gamma} \leq p(x,y) \leq \gamma n^{\alpha_p},
\end{displaymath}
i.e., the parameters are polynomially bounded by the input size $n$ with $\alpha_p$ as exponent.
A parameter setting $\mathbb{\alpha}$ is called \emph{trivial}, if $\lcsy{\alpha}$ has finite many instances for all $\gamma \geq 1$.
\end{definition}
%
To give an example, the LCS instances $x=a^{k}b^{k}$ and $y = ba^k$ for $k\geq 1$ are in $\lcsy{\mathbb{\alpha}}$ for all $\gamma \geq 4$, with the parameter setting $\mathbb{\alpha}$ being $\alpha_m = \alpha_L = \alpha_\Delta = 1$, $\alpha_\delta = 0$, $\alpha_M = 2$ and $\alpha_d = 1$.\todo{Example, maybe unnecessary?}


\begin{table}[t]
\centering
%\setlength{\tabcolsep}{10pt}
%\renewcommand{\arraystretch}{1.25}
\begin{tabular}{@{}ll@{}}
\toprule
\textbf{Parameter} & \textbf{Restriction} \\
\midrule
\(m\) & \(0 \le \alpha_m \le 1\) \\ \midrule
\(L\) & \(0 \le \alpha_L \le \alpha_m\) \\ \midrule
\(\delta\) &
\(
\begin{cases}
0 \le \alpha_\delta \le \alpha_m & \text{if } \alpha_L = \alpha_m,\\
\alpha_\delta = \alpha_m          & \text{otherwise}
\end{cases}
\) \\ \midrule
\(\Delta\) &
\(
\begin{cases}
\alpha_\delta \le \alpha_\Delta \le 1 & \text{if } \alpha_L=\alpha_m=1,\\
\alpha_\Delta = 1                     & \text{otherwise}
\end{cases}
\) \\ \midrule
\(|\Sigma|\) & \(0 \le \alpha_{\Sigma} \le \alpha_m\) \\ \midrule
\(d\) & \(\max\{\alpha_L,\alpha_{\Sigma}\} \le \alpha_d \le
\min\{\,2\alpha_L+\alpha_{\Sigma},\,\alpha_L+\alpha_m,\,\alpha_L+\alpha_\Delta\,\}\) \\ \midrule
\(M\) &
\(
\max\{1,\alpha_d,\,2\alpha_L-\alpha_{\Sigma}\} \le \alpha_M \le \alpha_L+1
\) \\
\bottomrule
\end{tabular}
\caption{Full set of parameter relations. Every parameter setting satisfying these relations is non-trivial.}
\label{tab:restrictions}
\end{table}


The full set of parameter relations is given in \autoref{tab:restrictions}.
Here we can see some trivial bounds, like $\alpha_L \leq \alpha m \leq \alpha_n$, which is due to the trivial relations $L \leq m \leq n$ which are true for every \lcs{} instance.
All of the relations are proven in the original paper \cite[section 6]{Bringman.2018}, but in this summary we will only proof the relations $2\alpha_L - \alpha_\Sigma \leq \alpha_M$.
Note that therefore it suffices to prove $\frac{L^2}{|\Sigma|} \leq M$.

\begin{theorem}
\label{thm:matching_pairs_lb}
For every \lcs{} instance the number of matching pairs is bounded by
\begin{displaymath}
	\dfrac{L^2}{|\Sigma|} \leq M .
\end{displaymath}
\end{theorem}

\begin{proof}
Per definition of matching pairs, we have $M = \sum_{\sigma \in \Sigma} \numSymbols{\sigma}{x} \numSymbols{\sigma}{y}$.
We can lower bound this by replacing $x$ and $y$ with one of their \lcs{} $z$.
Hence, $M \geq \sum_{\sigma \in \Sigma} \numSymbols{\sigma}{z}^2$.\todo{finish proof}
\end{proof}




