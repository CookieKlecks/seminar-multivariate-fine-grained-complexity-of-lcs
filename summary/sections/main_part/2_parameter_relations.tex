\section{Find Parameter Relations}
\label{sec:fin_param_rels}
The goal of this step is to find all valid and non-trivial parameter combinations.
With this we can restrict the actual parameter space and hence the space to span by the reductions.
Before finding the actual relations, we first define a so called \emph{parameter setting} and its non-triviality.

\begin{definition}[Parameter Setting]
Let $\mathcal{P}$ be the set of parameters.
We define $\mathbb{\alpha} = (\alpha_p)_{p \in \mathcal{P}}$ as a \emph{parameter setting}.
We further define $\lcsy{\mathbb{\alpha}}$ as the subset of \lcs{} instances, where %the input strings $x$ and $y$ fulfill
\begin{displaymath}
	\forall p \in \mathcal{P}: \  \frac{n^{\alpha_p}}{\gamma} \leq p(x,y) \leq \gamma n^{\alpha_p}.
\end{displaymath}
%i.e., the parameters are polynomially bounded by the input size $n$ with $\alpha_p$ as exponent.
A parameter setting $\mathbb{\alpha}$ is called \emph{trivial}, if $\lcsy{\alpha}$ has finite many instances for all $\gamma \geq 1$.
\end{definition}
%
%To give an example, the LCS instances $x=a^{k}b^{k}$ and $y = ba^k$ for $k\geq 1$ are in $\lcsy{\mathbb{\alpha}}$ for all $\gamma \geq 4$, with the parameter setting $\mathbb{\alpha}$ being $\alpha_m = \alpha_L = \alpha_\Delta = 1$, $\alpha_\delta = 0$, $\alpha_M = 2$ and $\alpha_d = 1$.\todo{Example, maybe unnecessary?}


\begin{table}[t]
\centering
%\setlength{\tabcolsep}{10pt}
%\renewcommand{\arraystretch}{1.25}
\begin{tabular}{@{}ll@{}}
\toprule
\textbf{Parameter} & \textbf{Restriction} \\
\midrule
\(m\) & \(0 \le \alpha_m \le 1\) \\ \midrule
\(L\) & \(0 \le \alpha_L \le \alpha_m\) \\ \midrule
\(\delta\) &
\(
\begin{cases}
0 \le \alpha_\delta \le \alpha_m & \text{if } \alpha_L = \alpha_m,\\
\alpha_\delta = \alpha_m          & \text{otherwise}
\end{cases}
\) \\ \midrule
\(\Delta\) &
\(
\begin{cases}
\alpha_\delta \le \alpha_\Delta \le 1 & \text{if } \alpha_L=\alpha_m=1,\\
\alpha_\Delta = 1                     & \text{otherwise}
\end{cases}
\) \\ \midrule
\(|\Sigma|\) & \(0 \le \alpha_{\Sigma} \le \alpha_m\) \\ \midrule
\(d\) & \(\max\{\alpha_L,\alpha_{\Sigma}\} \le \alpha_d \le
\min\{\,2\alpha_L+\alpha_{\Sigma},\,\alpha_L+\alpha_m,\,\alpha_L+\alpha_\Delta\,\}\) \\ \midrule
\(M\) &
\(
\max\{1,\alpha_d,\,2\alpha_L-\alpha_{\Sigma}\} \le \alpha_M \le \alpha_L+1
\) \\
\bottomrule
\end{tabular}
\caption{Full set of parameter relations. Every parameter setting satisfying these relations is non-trivial.}
\label{tab:restrictions}
\end{table}


The full set of parameter relations is given in \autoref{tab:restrictions}.
Here we can see some trivial bounds, like $\alpha_L \leq \alpha_m \leq \alpha_n$, which is due to the relations $L \leq m \leq n$ true for every \lcs{} instance.
All of the relations are proven in the original paper \cite[section 6]{Bringman.2018}, but in this summary we will only show $\frac{L^2}{|\Sigma|} \leq M$ which implies $2\alpha_L - \alpha_\Sigma \leq \alpha_M$.

\begin{theorem}
\label{thm:matching_pairs_lb}
For every \lcs{} instance the number of matching pairs is bounded by
\(
	M \geq \dfrac{L^2}{|\Sigma|}.
\)
\end{theorem}

\begin{proof}
Per definition of matching pairs, we have $M = \sum_{\sigma \in \Sigma} \numSymbols{\sigma}{x} \numSymbols{\sigma}{y}$.
With $z$ being an \lcs{} of $x$ and $y$, we obtain $M \geq \sum_{\sigma \in \Sigma} \numSymbols{\sigma}{z}^2$.
Further $L = \sum_{\sigma \in \Sigma} \numSymbols{\sigma}{z}$. 
With the arithmetic-quadratic mean inequality we have $M / |\Sigma| \geq \left(L/|\Sigma|\right)^2$ and the claim follows.
\end{proof}




