\section{Introduction}
Finding longest common subsequences (LCS) in strings is a well-studied field with applications running from file comparisons to DNA sequence analysis\todo{add citations}.
The task is to find for two strings $x$ and $y$ over the alphabet $\Sigma$ a common subsequence $z$ of maximal length.
The problem can be solved with dynamic programming in $\bigO{n^2}$ where $n$ is the length of the input strings.
Only a logarithmic improvement to this is known\todo{cite} and a fine-grained analysis showed that polynomial improvements would refute the Strong Exponential Time Hypothesis\todo{cite}.

However there are still new algorithms being developed.
They are often inspired by real-world applications where we can observe specific properties of the strings.
For example for file comparisons it is often the case that $x$ and $y$ only differ at few positions.
That enables to develop algorithms that exploit this property to achieve faster running times for these instances.
This motivates a multivariate fine-grained complexity analysis of LCS where the goal is to find lower bounds for the running times depending on a set of parameters, not only the input length $n$.

In the following we will use the notation $\numSymbols{\sigma}{x}$ for the number of occurrences of the symbol $\sigma$ in the string $x$.


\subsection{Main Results}


\subsection{Related Work}

\autoref{tab:related_work} shows an overview of the related work on multivariate LCS.

\begin{table}[hb]
\centering
\begin{tabular}{@{}p{0.48\linewidth}p{0.45\linewidth}@{}}
\toprule
\textbf{Reference} & \textbf{Running Time} \\
\midrule
Wagner and Fischer~[84]\todo{add citations} & $\mathcal{O}(mn)$ \\
Hunt and Szymanski~[47] & $\mathcal{O}((n+M)\log n)$ \\
Hirschberg~[45] & $\mathcal{O}(n\log n + Ln)$ \\
Hirschberg~[45] & $\mathcal{O}(n\log n + L\delta\log n)$ \\
Masek and Paterson~[65] & $\mathcal{O}(n + nm/\log^{2} n)\ \text{assuming } |\Sigma|=\mathcal{O}(1)$ \\
& $\mathcal{O}\!\left(n + nm\cdot(\frac{\log\log n}{\log n})^{2}\right)$\footnote{todo} \\
Nakatsu, Kambayashi and Yajima~[71] & $\mathcal{O}(n\delta)$ \\
Apostolico~[13] & $\mathcal{O}\!\left(n\log n + d\log\!\left(\tfrac{mn}{d}\right)\right)$ \\
Myers~[70] & $\mathcal{O}(n\log n + \Delta^{2})$ \\
Apostolico and Guerra~[14] & $\mathcal{O}\!\left(n\log n + Lm\min\{\log m,\log(n/m)\}\right)$ \\
Wu, Manber, Myers and Miller~[87] & $\mathcal{O}(n\log n + \delta\Delta)^{\;3}$ \\
Eppstein, Galil, Giancarlo and Italiano~[37] & $\mathcal{O}\!\left(n\log n + d\log\log\min\{d, nm/d\}\right)$ \\
Iliopoulos and Rahman~[50] & $\mathcal{O}(n + M\log\log n)$ \\
\bottomrule
\end{tabular}
\vspace{1em}
\caption{Taken from \cite{Bringman.2018}. Overview of selected LCS algorithms. The parameters are defined in \autoref{sec:def-params}.}
\label{tab:related_work}
\end{table}




\subsection{Hardness Hypothesis}
As basis for the reductions the \emph{Orthogonal Vectors} (OV) problem is used.
Here two sets $\mathcal{A}, \mathcal{B} \subseteq \{0,1\}^D$ are given and the task is to decide whether there are $a \in \mathcal{A}, b \in \mathcal{B}$ s.t. $\langle a,b \rangle := \sum_{i=1}^D a[i]\cdot b[i] = 0$.
Based on this problem, the authors of the original paper proof the lower bounds based on the following hypothesis.
%
\begin{definition}[Unbalanced Orthogonal Vectors Hypothesis (UOVH)]
For any $\alpha, \beta \in (0,1]$, the following problem requires time $n^{\alpha + \beta - o(1)}$:
Given a number $n$, solve a given OV instance with 
$D = n^{o(1)}$ and $|\mathcal{A}| = \bigO{n^\alpha}$ and $|\mathcal{B}| = \bigO{n^\beta}$.
\end{definition}
%
They prove that this hypothesis is equivalent to the \emph{Orthogonal Vectors Hypothesis} (see \cite[Lemma 5.1]{Bringman.2018}), which itself implies the \emph{Strong Exponential Time Hypothesis}.


%\subsection{Technical Tools}
%In this section we state lemmata from the original paper that are needed for later proofs.
%Due to size constraints, we do not prove them here.
%
%\begin{lemma}[Greedy Prefix Matching]\footnote{Copied from \cite[Lemma 7.1]{Bringman.2018}}
%\label{lem:greedy_prefix_match}
%For any strings $w, x, y$, we have
%\(
%L(wx, wy) = |w| + L(x, y)
%\)
%and
%\(
%d(wx, wy) = |w| + d(x, y).
%\)
%\end{lemma}
%
%\begin{lemma}\footnote{Copied from \cite[Lemma 7.6]{Bringman.2018}}
%\label{lem:guards}
%For any strings $x, y, z$ and $\ell \geq \numSymbols{0}{x} + |z|$ we have
%\(
%L\left(x0^{\ell}y,\,0^{\ell}z\right) = \ell + L\left(0^{\#_{0}(x)}y,\,z\right).
%\)
%\end{lemma}


