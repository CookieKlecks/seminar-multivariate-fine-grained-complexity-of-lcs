\section{Introduction}
Finding longest common subsequences (\lcs{}) in strings is a well-studied field with applications ranging from file comparisons~\cite{Hunt.1976,Miller.1985} to DNA sequence analysis~\cite{Altschul.1990}.
The task is to find for two strings $x$ and $y$ of length $n$ a common subsequence $z$ of maximal length.
\lcs{} can be solved with dynamic programming in $\bigO{n^2}$. 
A fine-grained analysis showed that polynomial improvements would refute the \emph{Strong Exponential Time Hypothesis} (SETH)\cite{Abboud.28.01.2015,Bringmann.2015}.

However, there are still new algorithms being developed.
They are often inspired by real-world applications where we can observe specific properties of the strings.
For example for file comparisons it is often the case that $x$ and $y$ only differ at few positions.
That enables to develop algorithms that exploit this property to achieve faster running times for these instances.
This motivates a multivariate fine-grained complexity analysis of LCS where the goal is to find lower bounds for the running times depending on a set of parameters, not only the input length $n$.
%
In the following we will use the notation $\numSymbols{\sigma}{x}$ for the number of occurrences of the symbol $\sigma$ in the string $x$.


\subsection{Main Results}
The authors provide a complete multivariate fine-grained complexity classification of LCS. 
They show that for all non-trivial parameter settings, the SETH-based conditional lower bounds obtained match the best known algorithms up to subpolynomial factors.
They prove that for alphabets of size $|\Sigma|\geq 3$, the optimal running time is $(n + \min\{d, \delta \Delta, \delta m\})^{1 - o(1)}$, while for binary alphabets the lower bound weakens\footnote{It is weaker because $\frac{\delta M}{n}$ is at most $\delta m$, because of the trivial bound $M \leq nm$.} to $(n + \min\{d, \delta \Delta, \tfrac{\delta M}{n}\})^{1-o(1)}$. 
In addition, they give a matching algorithm for $|\Sigma|=2$ with running time $\bigO(n + \tfrac{\delta M}{n})$, which constitutes the first polynomial improvement for a special case of LCS since 1990.
%These results show that polynomial improvements beyond these bounds would contradict SETH, and the results reveal a genuine complexity gap between binary and larger alphabets.
In this summary, we will only discuss the case $|\Sigma| \geq 3$.

\subsection{Related Work}
See \autoref{tab:related_work} for an overview of different \lcs{} algorithms on multivariate LCS progressing through time.
They either use dynamic programming approaches \cite{Wagner.1974,Hunt.1977,Hirschberg.1977,Masek.1980,Nakatsu.1982,Apostolico.1986,Apostolico.1987,Eppstein.1992,Iliopoulos.2009} or shortest paths on the edit graph of the input strings \cite{Myers.1986,Wu.1990}.
Note that the progress was restricted to log-factor improvements since 1990 \cite{Wu.1990}.
This motivates a fine-grained complexity analysis as it seems unlikely to find further polynomial improvements.

\begin{table}[ht]
\centering
\begin{tabular}{@{}p{0.48\linewidth}p{0.45\linewidth}@{}}
\toprule
\textbf{Reference} & \textbf{Running Time} \\
\midrule
Wagner and Fischer, 1974 \cite{Wagner.1974} & $\mathcal{O}(mn)$ \\
Hunt and Szymanski, 1977 \cite{Hunt.1977} & $\mathcal{O}((n+M)\log n)$ \\
Hirschberg, 1977 \cite{Hirschberg.1977} & $\mathcal{O}(n\log n + Ln)$ \\
Hirschberg, 1977 \cite{Hirschberg.1977} & $\mathcal{O}(n\log n + L\delta\log n)$ \\
Masek and Paterson, 1980 \cite{Masek.1980} & $\mathcal{O}(n + nm/\log^{2} n)\ \text{assuming } |\Sigma|=\mathcal{O}(1)$ \\
& $\mathcal{O}\!\left(n + nm\cdot(\frac{\log\log n}{\log n})^{2}\right)$\footnote{See \cite{Bille.2008} for the extension to non-constant alphabet}\\
Nakatsu, Kambayashi and Yajima, 1982 \cite{Nakatsu.1982} & $\mathcal{O}(n\delta)$ \\
Apostolico, 1986 \cite{Apostolico.1986} & $\mathcal{O}\!\left(n\log n + d\log\!\left(\tfrac{mn}{d}\right)\right)$ \\
Myers, 1986 \cite{Myers.1986} & $\mathcal{O}(n\log n + \Delta^{2})$ \\
Apostolico and Guerra, 1987 \cite{Apostolico.1987} & $\mathcal{O}\!\left(n\log n + Lm\min\{\log m,\log(n/m)\}\right)$ \\
Wu, Manber, Myers and Miller, 1990 \cite{Wu.1990} & $\mathcal{O}(n\log n + \delta\Delta)^{\;3}$ \\
Eppstein, Galil, Giancarlo and Italiano, 1992 \cite{Eppstein.1992} & $\mathcal{O}\!\left(n\log n + d\log\log\min\{d, nm/d\}\right)$ \\
Iliopoulos and Rahman, 2009 \cite{Iliopoulos.2009} & $\mathcal{O}(n + M\log\log n)$ \\
\bottomrule
\end{tabular}
\vspace{1em}
\caption{Taken from \cite{Bringman.2018}. Overview of selected LCS algorithms. The parameters are defined in \autoref{sec:def-params}.}
\label{tab:related_work}
\end{table}



\subsection{Hardness Hypothesis}
As basis for the reductions the \emph{Orthogonal Vectors} (OV) problem is used.
Here two sets $\mathcal{A}, \mathcal{B} \subseteq \{0,1\}^D$ are given and the task is to decide whether there are $a \in \mathcal{A}, b \in \mathcal{B}$ s.t. $\langle a,b \rangle := \sum_{i=1}^D a[i]\cdot b[i] = 0$.
Based on this problem, the authors of the original paper proof the lower bounds based on the following hypothesis.
%
\begin{definition}[Unbalanced Orthogonal Vectors Hypothesis (\uovh)]
For any $\alpha, \beta \in (0,1]$, the following problem requires time $n^{\alpha + \beta - o(1)}$:
Given a number $n$, solve a given OV instance with 
$D = n^{o(1)}$ and $|\mathcal{A}| = \bigO{n^\alpha}$ and $|\mathcal{B}| = \bigO{n^\beta}$.
\end{definition}
%
\uovh{} is equivalent to the \emph{Orthogonal Vectors Hypothesis} \cite[Lemma 5.1]{Bringman.2018} that implies SETH.


%\subsection{Technical Tools}
%In this section we state lemmata from the original paper that are needed for later proofs.
%Due to size constraints, we do not prove them here.
%
%\begin{lemma}[Greedy Prefix Matching]\footnote{Copied from \cite[Lemma 7.1]{Bringman.2018}}
%\label{lem:greedy_prefix_match}
%For any strings $w, x, y$, we have
%\(
%L(wx, wy) = |w| + L(x, y)
%\)
%and
%\(
%d(wx, wy) = |w| + d(x, y).
%\)
%\end{lemma}
%
%\begin{lemma}\footnote{Copied from \cite[Lemma 7.6]{Bringman.2018}}
%\label{lem:guards}
%For any strings $x, y, z$ and $\ell \geq \numSymbols{0}{x} + |z|$ we have
%\(
%L\left(x0^{\ell}y,\,0^{\ell}z\right) = \ell + L\left(0^{\#_{0}(x)}y,\,z\right).
%\)
%\end{lemma}


