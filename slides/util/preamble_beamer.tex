\newlength{\nodeDistance}\setlength{\nodeDistance}{1.5cm} % default distance between two nodes
\newlength{\partitionRadius}\setlength{\partitionRadius}{0.4cm} % default radius of partition line

\tikzstyle{highlight} = [color=kit-red80]
\tikzstyle{dim} = [thin, color=kit-gray40]
\tikzstyle{tight} = [inner sep = 0.2ex]

\tikzstyle{node} = [draw,circle]
\tikzstyle{edge} = [draw, inner sep = 0.2ex]
\tikzstyle{cut edge} = []

\tikzstyle{color A} = [color=kit-orange100]
\tikzstyle{color B} = [color=kit-lightgreen100]
\tikzstyle{color C} = [color=kit-cyan100]
\tikzstyle{color D} = [color=kit-purple80]

\tikzstyle{partition} = [dashed, thick]
\tikzstyle{partition A} = [partition, color A]
\tikzstyle{partition B} = [partition, color B]
\tikzstyle{partition C} = [partition, color C]

% Ident algorithm2e code. Necessary because probably \parident is not defined
\setlength{\algomargin}{2ex}


% Using overlays in TikZ pictures.
% Inspired from https://tex.stackexchange.com/a/55849

% Keys to support piece-wise uncovering of elements in TikZ pictures:
% \node[visible on=<2->](foo){Foo}
% \node[visible on=<{2,4}>](bar){Bar}   % put braces around comma expressions
%
% Internally works by setting opacity=0 when invisible, which has the 
% adavantage (compared to \node<2->(foo){Foo} that the node is always there, hence
% always consumes space plus that coordinate (foo) is always available.
%
% The actual command that implements the invisibility can be overriden
% by altering the style invisible. For instance \tikzsset{invisible/.style={opacity=0.2}}
% would dim the "invisible" parts. Alternatively, the color might be set to white, if the
% output driver does not support transparencies (e.g., PS) 
%
\tikzset{
  invisible/.style={opacity=0},
  invisible text/.style={text opacity=0},
  %
  visible on/.style={alt={#1}{}{invisible}},
  text visible on/.style={alt={#1}{}{invisible text}},
  %
  alt/.code n args={3}{%
	\alt#1{\pgfkeysalso{#2}}{\pgfkeysalso{#3}} % \pgfkeysalso doesn't change the path
  },
}