\begin{frame}
	\frametitle{Non-Triviality of Parameter Space}
	\framesubtitle{Definitions}

	\begin{definition}[Parameter Setting]
		\begin{itemize}
			\setlength{\itemindent}{1em}
			\item Set of parameters $\mathcal{P}$,\quad $\mathbf{\alpha} := \alpha_{p \in \mathcal{P}}$
			\pause
			\item $\lcsy{\mathbf{\alpha}}$ defined as problem of computing LCS of $x$ and $y$, where:
			\begin{displaymath}
				\frac{n^{\alpha_p}}{\gamma} \leq p(x,y) \leq n^{\alpha_p}\gamma, \quad \text{for all } p \in \mathcal{P}
			\end{displaymath}
			\pause
			\item Parameter setting $\mathbf{\alpha}$ is \emph{trivial} if for all $\gamma \geq 1$: $\lcsy{\mathbf{\alpha}}$ has finitely many instances.
		\end{itemize}
	\end{definition}
	\pause
	
	\begin{alertblock}{Todo:}
		Show that every parameter setting within bounds is non-trivial, i.e., has infinitely many instances.
	\end{alertblock}
	
	\begin{example}
		$x := 1^{\Delta + 1}$ and $y := 1$ proves non-triviality for parameter $\Delta$.
	\end{example}
\end{frame}